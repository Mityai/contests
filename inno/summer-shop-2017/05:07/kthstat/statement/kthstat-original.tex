%
% Летние всероссийские школьные учебно-тренировочные сборы по информатике
%   14 июня 2011 года
%
% Условие: Сергей Копелиович
% Тесты: Сергей Федоров
% Решения: Роман Андреев, Сергей Копелиович
% Идея: Сергей Копелиович
%

\begin{problem}{Опять k-я статистика}
{kthstat.in}{kthstat.out}
{2 секунды}{256 мебибайт}{}

Изначально вам дан массив целых чисел.

Нужно уметь отвечать на три запроса:

\begin{itemize}
  \item{\t{+ i x} "--- Вставить на $i$-ю позицию число $x$
        (размер массива увеличивается на $1$)}
  \item{\t{- i} "--- Удалить число на $i$-й позиции
        (размер массива уменьшается на $1$)}
  \item{\t{? L R x} "--- Сказать, сколько чисел $y$
        на позициях $L \le i \le R$ таких, что $y \le x$ ($|x| \le 10^9$)}
\end{itemize}

Общаться с системой вы будете с помощью следующих функций:
\begin{center}
  \t{Start~(n,~a)} \\
  \t{Add~(i,~x)} \\
  \t{Del~(i)} \\
  \t{Get~(L,~R,~x)}
\end{center}

Все индексы $i$, $L$, $R$ нумеруются с нуля.
Все числа в запросах целые.
Все запросы корректны.

Например, \t{Add~(0,~x)} означает добавление $x$ в начало массива.

В функции \t{Start} вы получаете исходный массив.
Она будет запускаться ровно один раз, до всех остальных функций.
В функциях \t{Add}, \t{Del}, \t{Get} вы отвечаете на соответствующие запросы.

Исходное число элементов в массиве "--- $0 \le N \le 10^5$,
числа в массиве по модулю не превышают $10^9$.

Число запросов "--- $1 \le K \le 10^5$.

\Example

\begin{example}
\exmp{
10
455184306 359222813 948543704 914773487 861885581 253523 770029097 193773919 581789266 457415808
- 1
? 2 5 527021001
? 0 5 490779085
? 0 5 722862778
+ 9 448694272
- 5
? 1 2 285404014
- 4
? 3 4 993634734
+ 0 414639071
}{%
1
2
2
0
2
}%
\end{example}

\end{problem}
