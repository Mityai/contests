% Problem author: Sergey Kopeliovich
% Original source: Харьковские сборы, февраль 2012

\begin{problem}{Точки в полуплоскости}{semiplane.in}{semiplane.out}{2 секунды}{256 мегабайт}{}

Есть $N$ точек на плоскости.
Точки равномерно распределены внутри квадрата $[0..C] \times [0..C]$.
Вам нужно научиться отвечать на запрос ``сколько точек лежит в полуплоскости''?

\InputFile

Число точек $N$ ($1 \le N \le 5 \cdot 10^4$), число запросов $M$ ($1 \le M \le 5 \cdot 10^4$),
константа $C$ (целое число от $1$ до $10^4$).
Далее $N$ точек $(X,Y)$ с целочисленными координатами.
Далее $M$ полуплоскостей $(a,b,c)$. Числа $a$, $b$, $c$ --- целые, по модулю не превосходят $10^4$. $a^2 + b^2 \not= 0$.
Считается, что точка лежит в полуплоскости тогда и только тогда, когда $ax + by + c \ge 0$.

\OutputFile

Для каждого из $M$ запросов одно целое число --- количество точек в полуплоскости.

\Example

\begin{example}
\exmp{
3 4 10
5 5
1 7
7 4
1 1 -9
1 1 -10
1 1 -11
1 1 -12
}{
2
2
1
0
}%
\end{example}

\end{problem}
