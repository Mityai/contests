\gdef\thisproblemauthor{Sergey Kopeliovich}
\gdef\thisproblemdeveloper{Sergey Kopeliovich}

\begin{problem}{Bridge Building}
{bridge.in}{bridge.out}
{5 seconds \textsl{(8 seconds for Java)}}{512 mebibytes}{} % 512 is important!

% Author in 2009: Vitaliy Aksenov
% Russian statement in 2009: Pavel Mavrin
% Limitations are increased 100 times by Sergey Kopeliovich

\epigraph{
  A long time ago in 2009...
}

In the village Zaykino, heavy rain is common.
After such rain, the river Dubrovka which can usually
be just stepped over, overflows.
To be able to get across the flooded river, the villagers want to build
a floating bridge.
Luckily, after construction of a bath-house which belongs
to a businessman who settled nearby, there are some logs left.

%В деревне Зайкино регулярно идут проливные дожди, в результате чего
%речка Дубровка, которую обычно можно просто перешагнуть, выходит из берегов.
%Чтобы можно было перейти разлившуюся реку, планируется построить
%плавучий мост из брёвен, оставшихся от строительства
%бани бизнесмена, поселившегося неподалёку.

All remaining logs have the same thickness.
There are $x$ logs of length $a$ and $y$ logs of length $b$.

%Все оставшиеся брёвна имеют одинаковую толщину.
%При этом есть $x$ брёвен длины $a$ и $y$ брёвен длины $b$.

The bridge will consist of $l$ rows,
each of which will be composed of one or more logs.
Unfortunately, the last saw in Zaykino drowned in Dubrovka
during the previous overflow and disappeared, so the logs
can not be cut into pieces.

%Построенный мост должен состоять из $l$ рядов, каждый из которых
%составлен из одного или нескольких брёвен.
%Пилить брёвна нельзя, так как последняя пила утонула при разливе Дубровки.

The chief engineer wants to build a bridge of maximum possible width.
The width of a bridge is determined by the minimum width of a row of logs
in it.

%Главный инженер хочет построить мост максимальной возможной ширины.
%Ширина моста определяется по минимальной ширине ряда брёвен в нём.

For example, if the villagers want to build a bridge of seven rows,
and there are six logs of length $3$ and ten logs of length $2$,
then they can build a bridge of width $5$.

%Например, если нужно построить мост из семи рядов, и при этом есть
%шесть брёвен длины $3$ и десять брёвен длины $2$,
%то можно построить мост ширины $5$.

\begin{center}
\includegraphics{bridge.1}
\end{center}

\InputFile

Input contains one or more test cases.
Each test case consists of five positive integers
$x$, $a$, $y$, $b$ and $l$.
Each of these numbers does not exceed $500$.
The total number of logs in each test case is at least $l$.

%Ввод состоит из одного или нескольких тестовых случаев.
%Каждый тестовый случай состоит из пяти целых положительных чисел
%$x$, $a$, $y$, $b$ и $l$.
%Каждое число не превосходит $500$. 
%Общее количество брёвен в каждом тестовом случае не меньше $l$.

Let $d = \max (x, a, y, b, l)$.
It is guaranteed that the sum of $d$ over all the tests is at most $5000$.

%Обозначим $d = \max (x, a, y, b, l)$.
%Гарантируется, что сумма $d$ по всем тестам не превосходит $5000$.

\OutputFile

For each test case, print an integer on a separate line:
the maximum possible width of the bridge.

%Для каждого тестового случая на отдельной строке выведите одно число "---
%максимальную возможную ширину моста.

\Example

\begin{example}
\exmpfile{01.t}{01.a.t}%
\end{example}

\end{problem}
