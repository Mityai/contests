\begin{problem}{Сплоченная команда}{team.in}{team.out}{1 секунда}{64 мегабайта}{}

% Идея: Екатерина Гладышева, Марина Мухачева
% Текст: Екатерина Гладышева

Скоро в ЛКШ планируется провести игру <<Форт Баярд>>. В ней могут принимать участие команды с числом игроков 
от $L$ до $R$, причем, если один ЛКШонок играет в некоторой команде, 
то в этой же команде обязательно должны играть все его друзья. Будем называть количество пар друзей в составе
команды ее \emph{сплоченностью}. Помогите школьникам собрать как можно более сплоченную команду.

\InputFile

В первой строке входного файла через пробел записаны четыре натуральных числа: $N$, $M$, $L$ и $R$ ($1 \le N \le 2000$,
$0 \le M \le 10^5$, $1 \le L \le R \le N$), где $N$~--- количество школьников, готовых принять участие в игре, а 
$M$~--- количество пар друзей. В каждой из следующих $M$ строк записана пара чисел $a_i, b_i$ ($1 \le a_i, b_i \le N$, $a_i \neq b_i$), означающяя, что
ЛКШата с номерами $a_i$ и $b_i$ являются друзьями.

\OutputFile

В первой строке выведите количество человек в искомой команде. Во второй строке через пробел запишите номера
всех игроков, входящих в ее состав. Если команду собрать нельзя, в единственной строке выходного файла
выведите \texttt{-1}.

\Examples

\begin{example}
\exmp{
12 13 7 9
1 3
1 2
2 3
3 4
5 6
6 8
6 7
5 7
5 8
7 8
11 9
9 12
10 9
}{
8
1 2 3 4 5 6 7 8
}%
\exmp{
7 6 5 6
1 4
4 6
3 7
2 3
2 7
5 7
}{
-1
}%
\end{example}

\end{problem}
