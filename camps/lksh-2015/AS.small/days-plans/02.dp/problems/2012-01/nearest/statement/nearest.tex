\begin{problem}{Наилучшее приближение}{nearest.in}{nearest.out}{1 секунда}{1 мегабайт (20 для java)}

Вам даны $N$ целых чисел. Ваша задача --- вставить ровно по одному знаку
``\texttt{+}'' или ``\texttt{-}'' между каждой парой соседних таким образом,
чтобы сделать значение получившегося выражения максимально близким
к заданному числу $A$.

\InputFile

Первая строка входного файла содержит два целых числа: $N$ ($1 \le N\le 10\,000$)
и $A$, которое по модулю не превосходит $10\,000$.
Далее следуют $N$ строк, в каждой из которых содержится ровно одно целое число 
$X_i$, не превосходящее по модулю $10\,000$. Кроме того, гарантируется, что сумма
абсолютных величин всех $N$ чисел также не превосходит $10\,000$.

\OutputFile

В первой строке необходимо вывести значение получившегося выражения
(которое должно быть настолько близко к $A$, насколько это возможно).
Во второй строке необходимо вывести само выражение, дающее такое значение,
в форме $X_1[+|-]X_2[+|-]\ldots X_{N-1}[+|-]X_N$. Если 
оптимальных решений несколько,
то разрешается выводить любое из них.

\Example

\begin{example}
\exmp{
3 0
3
-2
1
}{
0
3+-2-1
}%
\end{example}

\end{problem}
