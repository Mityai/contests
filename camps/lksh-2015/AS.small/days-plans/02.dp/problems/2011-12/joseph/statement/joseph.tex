\begin{problem}{Joseph Problem}{joseph.in}{joseph.out}{2 секунды}{64 мегабайт}{}

$N$ мальчиков стоят по кругу. Они начинают считать себя по часовой стрелке, счет ведется с единицы.
Как только количество посчитанных достигает $p$, последний посчитанный ($p$-й) мальчик покидает круг,
а процесс счета начинается со следующего за ним мальчика и вновь ведется с единицы.

Последний оставшийся в кругу выигрывает.

Можете ли вы посчитать, номер выигрывшего мальчика в исходном кругу? (мальчики нумеруются 
числами от $1$ до $N$ по часовой стрелке, начиная
с того самого мальчика, с которого начинался счет).

\InputFile

Во входном файле два целых числа --- $N$ и $P$ ($1 \le N, P \le 10^6$).

\OutputFile

Выведите номер выигрывшего мальчика.

\Example

\begin{example}
\exmp{
3 4
}{
2
}%
\end{example}

\end{problem}
