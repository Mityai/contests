\begin{problem}{Мостостроение}{bridge.in}{bridge.out}{2 секунды}{256 мегабайт}

% src: http://neerc.ifmo.ru/school/archive/2009-2010/ru-olymp-team-spb-2009-statements.pdf

% Автор задачи: Виталий Аксенов
% Автор условия: Павел Маврин

В деревне Гадюкино регулярно идут проливные дожди, в результате
чего речка Вонючка, которую обычно можно просто перешагнуть, выходит
из берегов. Чтобы можно было перейти разлившуюся реку, планируется 
построить плавучий мост из бревен, оставшихся от строительства
бани бизнесмена, поселившегося неподалеку.

Все оставшиеся бревна имеют одинаковую толщину. При этом есть $x$ бревен
длины $a$ и $y$ бревен длины~$b$.
                            
Построенный мост должен состоять из $l$ рядов, каждый из которых
составлен из одного или нескольких бревен. Пилить бревна нельзя,
так как последняя пила утонула при разливе Вонючки.

Главный инженер хочет построить мост максимальной возможной
ширины, при этом ширина моста определяется по минимальной ширине
ряда бревен.

Например, если нужно построить мост из семи рядов, и при этом есть
шесть бревен длины 3 и десять бревен длины 2, то можно
построить мост ширины 5.

\begin{center}
\includegraphics{bridge.1}
\end{center}

\InputFile

Входной файл содержит пять натуральных чисел: $x$, $a$, $y$, $b$ и $l$.
Все числа не превышают 150. Общее количество бревен не меньше $l$.

\OutputFile

Выведите в выходной файл одно число --- максимальную возможную ширину моста.

\Examples

\begin{example}%
\exmp{
6 3 10 2 7
}{
5
}%
\exmp{
10 7 20 9 25
}{
9
}%
\end{example}

\end{problem}
             
