% Source: http://acm.timus.ru/problem.aspx?space=1&num=1326

\begin{problem}{Крышки}{bottletaps.in}{bottletaps.out}{1 секундa}{256 мегабайт}

У программиста Петрова есть хобби "--- собирать крышки от пивных бутылок.
Ничего странного, он знает еще с сотню программистов, которые очень уважают пиво. Да, они тоже собирают крышки. Не все из них, конечно, но некоторые.
Если честно, то часть своих крышек он просто купил, уже без бутылок. Да, это не совсем спортивно, зато коллекция теперь более солидная.
Одна вот беда "--- не хватает ему для полноты коллекции еще нескольких редких крышек. Он даже нашел в Интернете программистов, которые согласны продать их ему.
Некоторые даже продают крышки сразу наборами, с большой скидкой. Почему продают, да еще со скидкой? А вы попробуйте объяснить жене, какая польза от пивных крышек. 
Она же не программист. Осталось выбрать оптимальные предложения. Если объяснить жене зачем надо хранить крышки еще возможно,
то почему на них надо тратить деньги "--- точно не поймет. Поэтому надо купить как можно дешевле.
Петров выписал на бумажку все варианты и задумался. Купить сразу все не получится, никаких денег не хватит.
Тогда надо купить самые необходимые, но подешевле. Да уж, без программы тут не обойдешься...

\InputFile

В первой строке записано число $N$ "--- количество недостающих Петрову крышек ($1 \le N \le 20$).
Далее идет $N$ строк "--- цена, за которую можно купить эту крышку, если покупать ее отдельно.
В следующей строке стоит число $M$ ($0 \le M \le 100$) "--- количество предложений по продаже наборов, содержащих нужные ему крышки.
Далее идет $M$ строк описывающих эти наборы.
В каждой строке первое число "--- цена набора, второе "--- количество крышек в наборе, далее перечислены номера крышек (каждый номер от $1$ до $N$),
которые в этот набор входят. Номера в наборе не повторяются. Все цены "--- положительные числа, не превосходящие $1000$.
В последней строке перечислен минимальный набор крышек, который Петров намерен купить в любом случае.

\OutputFile

Выведите минимальную сумму, необходимую Петрову, чтобы купить все крышки из приведённого в последней строке набора.

\Examples

\begin{example}
\exmp{4
10
11
12
13
3
17 2 1 3
25 3 2 3 4
15 2 3 4
3 1 3 4}{25
}%
\end{example}

\end{problem}
