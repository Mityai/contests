\documentclass[12pt,a4paper,oneside]{article}

\usepackage{cmap}
\usepackage[T2A]{fontenc}
\usepackage[utf8]{inputenc}
\usepackage[english,russian]{babel}
\usepackage[russian]{olymp}
\usepackage{graphicx}
\usepackage{amsmath,amssymb}
\usepackage{epigraph}
\usepackage[russian]{hyperref}
\usepackage{color}
\usepackage{lastpage}
\usepackage{import}
\usepackage{verbatim}

\renewcommand{\t}{\texttt}
\renewcommand{\le}{\leqslant}
\renewcommand{\ge}{\geqslant}

\binoppenalty=10000
\relpenalty=10000
\exhyphenpenalty=10000

\newcommand{\ProblemLabel}{undefined}
\newcommand{\ProblemTL}{undefined}
\newcommand{\ProblemML}{undefined}
\newcommand{\ProblemName}{undefined}

\def\O{\mathcal{O}}

\def\probl#1#2#3#4#5{
  \renewcommand{\ProblemName}{#2}
  \renewcommand{\ProblemLabel}{#3}
  \renewcommand{\ProblemTL}{#4}
  \renewcommand{\ProblemML}{#5}
  %\input #2.tex
  \import{../problems/#1/#2/statement/}{#2}
  %\input ../problems/#1/#2/statement/#2.tex
}
\def\problfix#1#2#3#4#5{
  \renewcommand{\ProblemName}{#2}
  \renewcommand{\ProblemLabel}{#3}
  \renewcommand{\ProblemTL}{#4}
  \renewcommand{\ProblemML}{#5}
  \input ../problems/#1/#2/statement/#2.tex
}
          
\newcommand{\Section}[1]{
  \hbox{\hspace{1em}}
  \vspace*{-2.5em}
  \section*{#1}
  \addcontentsline{toc}{section}{#1}
  \vspace*{-0.5em}
}

\contest
{ЛКШ.2015.Июль.AS.День 2: динамичный тур}%
{Судиславль, Берендеевы Поляны}%
{7 июля 2015, вторник}%

%\sectionfont{\fontsize{8}{8}\selectfont}

\definecolor{dkgreen}{rgb}{0,0.6,0}
\definecolor{brown}{rgb}{0.5,0.5,0}

\def\compact{
  \setlength{\parskip}{-5pt}
  \setlength{\itemsep}{5pt}
}

\begin{document}

\vspace{2em}
\tableofcontents

\vspace{1em}
\noindent \underline{\hbox to 1\textwidth{{ } \hfil{ } \hfil{ } }}

\pagebreak

\probl{2012-01}{nails}{A}{0.5 sec}{256 mb}      % - min по суммарной длине набор отрезков, покрывающий все точки на прямой
\probl{2011-10}{longpath}{B}{0.5 sec}{256 mb}   % - самый длинный путь в невзвешенном графе без циклов (N <= 10 000, M <= 100 000)
\probl{2011-12}{joseph}{C}{0.5 sec}{256 mb}     
\probl{2014-10}{sequence}{D}{0.5 sec}{256 mb}   % - наибольшая возрастающая подпоследовательность, отношение возрастания -- делимость
\probl{2014-10}{palindr}{E}{0.5 sec}{256 mb}    % - выбрать самую длинную подпоследовательность-палиндром за O(n^2).
\probl{2014-03}{folding}{F}{0.5 sec}{256 mb}    % - timus.1238 (запаковать строку), n <= 100
\probl{2011-10}{team}{G}{0.5 sec}{256 mb}       % - Найти компоненты связности, а по ним рюкзак
\probl{2012-01}{nearest}{H}{2 sec}{16 mb}     % - +a or -a. Нужно получить максимальную близкую к X сумму. N <= 10 000. Sum|a| <= 10 000.
\probl{2014-10}{ship}{I}{3 sec}{256 mb}       % - отправиль минимальным числом кораблей массив a[1:n], за один ход можно отправлять префикс + суффикс. (f,w)[L,R]. n <= 10^4.
\probl{2014-10}{bottletaps}{J}{1 sec}{256 mb} % - Набрать множество крышек, как сумму подмножеств.
\probl{2011-12}{bridge1}{K}{0.5 sec}{256 mb}     
\probl{2015-06}{bridge2}{L}{1 sec}{256 mb}     % - O(N^2) усложнение предыдущей задачи
\probl{2014-11}{casino}{M}{1 sec}{256 mb}     % - РОИ'2005, за ход можно удалить подстроку из данных
\probl{2014-10}{irreduce}{N}{0.5 sec}{256 mb}   % - количество неприводимых унитарных многочленов длины n над F_p

\end{document}
