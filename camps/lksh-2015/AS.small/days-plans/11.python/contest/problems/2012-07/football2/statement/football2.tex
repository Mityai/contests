\begin{problem}{Футбол}{football2.in}{football2.out}{2 секунды}{64 MB}{1D}

На футбольном поле размером
$x\times y$ находятся $n$ футболистов. Они уже очень устали
и стоят на месте, но ждут, куда упадет мяч, чтобы побежать к нему.
Футболист бежит к мячу в том случае, если мяч упал к этому
футболисту ближе, чем к любому другому футболисту. 
Требуется определить для каждого футболиста
границы зоны, при попадании
в которую он побежит к мячу, если известно, что она представляет собой многоугольник.

\InputFile

В первой строке входного файла заданы 
три 
целых числа $x$, $y$ и $n$ ($2\le x,y\le 10^5$, $1\le n\le 1\,000$).
Следующие $n$ строк содержат 
целые координаты футболистов $x_i$ $y_i$ ($0<x_i<x$, $0<y_i<y$). Никакие два
футболиста не стоят в одной точке.

\OutputFile

В выходной файл выведите $n$ строк. В каждой из строк первое число ---
количество вершин зоны $k_i$, далее $k_i$ чисел --- координаты вершин $x_{ij}$ $y_{ij}$
в порядке обхода против часовой стрелки, начиная с самой нижней из самых левых вершин
зоны. Вещественные числа выводите с максимальной точностью.

\Example

\begin{example}
\exmp{
4 4 4
1 1
1 3
3 1
3 3
}{
4 0 0 2 0 2 2 0 2
4 0 2 2 2 2 4 0 4
4 2 0 4 0 4 2 2 2
4 2 2 4 2 4 4 2 4
}%
\end{example}

\end{problem}