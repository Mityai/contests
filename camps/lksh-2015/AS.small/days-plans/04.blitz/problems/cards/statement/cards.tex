%Origin: 20111127 - PhML 30 Olympiad, District Tour
%Author: Ivan Kazmenko

\gdef\thisproblemauthor{Иван Казменко}
\begin{problem}{Карточки с буквами}
{\textsl{стандартный ввод}}{\textsl{стандартный вывод}}
%{cards.in}{cards.out}
{2 секунды}{256 мебибайт}{}
%{\textsl{В этой задаче нужно составить из карточек с буквами палиндром
%максимальной длины.}}

У девочки Маши есть $n$ карточек.
На каждой карточке написана какая-то буква латинского алфавита.
Маша хочет составить из своих карточек (возможно, не всех) палиндром,
то есть такую строку, которая бы читалась одинаково как справа налево,
так и слева направо.

К примеру, если у Маши четыре карточки, на которых написаны буквы
\t{A}, \t{B}, \t{A} и \t{C}, то из них можно составить, в частности,
палиндромы <<\t{C}>>, <<\t{AA}>>, <<\t{ABA}>>.
С другой стороны, строки <<\t{ABAC}>>, <<\t{BA}>>, <<\t{AABC}>> из этих
карточек составить тоже можно, но палиндромами они не являются.

Помогите Маше составить из своих карточек палиндром
максимально возможной длины.

\InputFile

В первой строке входного файла задано целое число $n$ ($1 \le n \le 100$) "---
количество карточек с буквами.
Во второй строке заданы сами буквы: эта строка состоит из $n$ символов,
каждый из которых является заглавной буквой латинского алфавита.
Учтите, что во второй строке нет пробелов, а завершается она символами
перевода строки (в системе Windows это два символа с ASCII-кодами 13 и 10).

\OutputFile

В первой строке выходного файла выведите палиндром максимальной длины,
который можно составить из машиных карточек.
Если палиндромов максимальной длины несколько, можно выводить любой из них.

\Examples

\begin{example}%
\exmp{
4
ABAC
}{
ABA
}%
\exmp{
6
AABBCC
}{
ABCCBA
}%
\exmp{
3
ABC
}{
A
}%
\end{example}

\Explanations

В первом примере максимальная длина палиндрома равна трём.
Правильным будет также ответ <<\t{ACA}>>.

Во втором примере удастся использовать все шесть карточек.
Существуют и другие правильные ответы, например,
<<\t{BCAACB}>>, <<\t{CBAABC}>>, \ldots

В третьем примере не получится сделать палиндром более чем из одной карточки.
Правильный ответ "--- любая из заданных букв.

%\PartialSolutions
%
%Решение, правильно работающее при ограничениях
%$1 \le n \le 10$,
%получит не менее $40$ баллов.

\end{problem}
