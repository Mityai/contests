%Origin: 20111127 - PhML 30 Olympiad, District Tour
%Author: Ivan Kazmenko

\gdef\thisproblemauthor{Иван Казменко}
\begin{problem}{Волшебные ночи}
{\textsl{стандартный ввод}}{\textsl{стандартный вывод}}
%{magic.in}{magic.out}
{2 секунды}{256 мебибайт}{}
%{\textsl{В этой задаче нужно найти количество волшебных ночей в году
%на далёкой планете.}}

Вокруг далёкой планеты Этан вращается три луны: Клементина, Лея и Матильда.
Каждую $k$-ю ночь наступает полнолуние Клементины,
каждую $l$-ю ночь "--- полнолуние Леи,
а каждую $m$-ю ночь "--- полнолуние Матильды.
В году на этой планете $n$ ночей, а Новый Год наступает днём.

Ночь на планете Этан считается волшебной, если в эту ночь наступает
полнолуние хотя бы у одной из лун.
Известно, что в последнюю ночь прошлого года полнолуние наступило
одновременно у всех трёх лун Этана.
Сколько волшебных ночей в текущем году?

\InputFile

В первой строке входного файла заданы четыре целых числа $k$, $l$, $m$ и $n$
($1 \le k, l, m, n \le 10^9$).
Числа разделены пробелами.

\OutputFile

В первой строке выходного файла выведите одно целое число "---
количество волшебных ночей в текущем году.

\Examples

\begin{example}%
\exmp{
3 4 5 10
}{
7
}%
\exmp{
5 5 5 10
}{
2
}%
\exmp{
30 29 31 360
}{
35
}%
\exmp{
2 4 6 5
}{
2
}%
\end{example}

\Explanations

В первом примере волшебными считаются
$3$-я, $4$-я, $5$-я, $6$-я, $8$-я, $9$-я и $10$-я ночи.

Во втором примере волшебных ночей только две "--- $5$-я и $10$-я ночи.

В третьем примере волшебными оказываются $12$ ночей, когда полнолуние
наступает у Клементины, $12$ ночей, когда полнолуние наступает у Леи,
и $11$ ночей, когда полнолуние наступает у Матильды.

В четвёртом примере во вторую ночь наступает полнолуние Клементины,
а в четвёртую "--- Клементины и Леи.
Поскольку в году всего пять ночей, следующее полнолуние Матильды случится
только в следующем году.

%\PartialSolutions
%
%Решение, правильно работающее при ограничениях
%$1 \le k, l, m, n \le 500$,
%получит не менее $40$ баллов.

\end{problem}
