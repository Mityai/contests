%
% Зимние всероссийские школьные учебно-тренировочные сборы по информатике
% понедельник, 3 июля 2012 года
% Источник: Дальневосточный четвертьфинал ACM ICPC NEERC 2011
%
\begin{problem}{Зал}
{\textsl{стандартный ввод}}{\textsl{стандартный вывод}}
%{тест}{ответ}
{2 секунды}{256 мебибайт}{}

Главный зал Института исследования неясных объектов в мутной воде имеет
форму длинного кордиора. Вдоль коридора расставлены $N$ аквариумов,
в которых выставлены различные морские твари. Аквариумы расставлены
на расстоянии $x_1, \ldots, x_N$ от входа в кордиор ($x_i < x_{i + 1}$).

Недавно институт обзавёлся новым руководством, которое приняло решение
убрать $M$ ($0 \le M \le N - 2$) аквариумов, так как их обслуживание
обходится слишком дорого.

Чтобы минимизировать ухудшение внешнего вида зала, было решено, что:
\begin{itemize}
  \item первый и последний аквариумы должны остаться на своих местах,
  \item максимальное расстояние между последовательными аквариумами
    должно быть как можно меньше.
\end{itemize}

Напишите программу, которая выберет аквариумы, которые нужно будет
убрать.

\InputFile

В первой строке записаны целые числа $N$ и $M$
($2 \le N \le 400$, $0 \le M \le N - 2$).
Во второй строке записаны $N$ целых чисел $x_i$ ($1 \le x_i \le 10^9$).

\OutputFile

Выведите единственное целое число "--- минимально возможное максимальное
расстояние между последовательными аквариумами.

\Example

\begin{example}
\exmp{
5 2
1 2 3 4 5
}{
2
}%
\end{example}

\end{problem}
