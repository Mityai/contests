%Origin: 201111?? - Leningrad Regional School Olympiad, District Tour
%Author: Natalya Ginzburg
%Description Author: Natalya Ginzburg
%Tests Author: Ivan Kazmenko
%Link: http://www.blog.republicofmath.com/archives/4228

\gdef\thisproblemauthor{Иван Казменко}
\begin{problem}{Треугольная таблица}
{\textsl{стандартный ввод}}{\textsl{стандартный вывод}}
%{table.in}{table.out}
{2 секунды}{256 мебибайт}{}

Рассмотрим следующую бесконечную треугольную таблицу чисел:

\begin{center}
\begin{tabular}{c c c c c c c c c c c}
~ & ~ & ~ & ~ & ~ & 1 & ~ & ~ & ~ & ~ & ~ \\
~ & ~ & ~ & ~ & 1 & ~ & 1 & ~ & ~ & ~ & ~ \\
~ & ~ & ~ & 1 & ~ & 2 & ~ & 1 & ~ & ~ & ~ \\
~ & ~ & 1 & ~ & 3 & ~ & 3 & ~ & 1 & ~ & ~ \\
~ & 1 & ~ & 4 & ~ & 5 & ~ & 4 & ~ & 1 & ~ \\
. & ~ & . & ~ & . & ~ & . & ~ & . & ~ & . \\
\end{tabular}
\end{center}

По краям в каждой строке стоят единицы, а остальные числа получаются
по <<правилу ромба>>: число снизу равно произведению двух чисел над ним
на предыдущей строке, увеличенному на $1$ и разделённому на число,
находящееся на две строки выше.

\begin{center}
\begin{tabular}{c c c}
~ & $C$ & ~ \\
$A$ & ~ & $B$ \\
~ & {\Large $\frac{A \times B + 1}{C}$} & ~ \\
\end{tabular}
\end{center}

Например, в пятой строке второе, третье и четвёртое числа равны соответственно
$\frac{1 \times 3 + 1}{1} = 4$,
$\frac{3 \times 3 + 1}{2} = \frac{10}{2} = 5$ и
$\frac{3 \times 1 + 1}{1} = 4$.

Может показаться удивительным, но, несмотря на деление,
все числа в таблице будут целыми.

По заданным $n$ и $k$ найдите $k$-е число в $n$-й строке этой таблицы.
Строки, а также числа в каждой строке, нумеруются с единицы.

\InputFile

В первой строке входного файла заданы два числа $n$ и $k$
($1 \le k \le n \le 1\,000\,000$).
Числа разделены одним пробелом.

\OutputFile

Выведите в выходной файл число, стоящее на $k$-м месте в $n$-й строке таблицы.

\Examples

\begin{example}%
\exmp{
5 2
}{
4
}%
\exmp{
6 3
}{
7
}%
\end{example}

\Explanations

В первом примере
$\frac{1 \times 3 + 1}{1} = 4$.

Во втором примере
$\frac{4 \times 5 + 1}{3} = \frac{21}{3} = 7$.

\end{problem}
