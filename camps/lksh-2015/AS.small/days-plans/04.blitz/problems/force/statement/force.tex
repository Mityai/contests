%
% Зимние всероссийские школьные учебно-тренировочные сборы по информатике
% понедельник, 3 июля 2012 года
% Источник: Дальневосточный четвертьфинал ACM ICPC NEERC 2011
%
\begin{problem}{Сила}
{\textsl{стандартный ввод}}{\textsl{стандартный вывод}}
%{тест}{ответ}
{2 секунды}{256 мебибайт}{}

Инженеры разработали новое подводное устройство. Для получения дополнительного
финансирования было решено назвать его <<наноустройством>>.

Задумавшись о том, как обосновать такое название, молодой специалист Вася
предложил измерять силу устройства в нано-ньютонах.

Предложение было с радостью принято, а вам было предложено написать
программное обеспечение для контроллера рулевого управления устройства.

У устройства есть четыре рулевых двигателя на левой, правой, верхней
и нижней сторонах. Их силы определяются целыми числами
$f_L$, $f_R$, $f_U$ и $f_D$, соответственно.

\emph{Допустимые} значениия этих сил лежат в пределах
от $-10^8$ до $10^8$ нано-ньютонов, включительно. Положительные значения
соответствуют движению вперёд, отрицательные "--- движению назад.

Несмотря на то, что все двигатели расположены параллельно, их всё ещё можно
использовать для поворота устройства.
Например, если $f_L = -10^8$, $f_R = 10^8$, $f_U = f_D = 0$, устройство
поворачивает влево.
Если $f_L = f_R = f_U = f_D = 10^8$, устройство двигается вперёд на полной
скорости.

Для человека, который управляет устройством, вместо установки значений
сил двигателей напрямую, удобнее установить \emph{полную силу} $T$,
горизонтальное отклонение $H$ и вертикальное отклонение $V$, которые
определяются следующим образом:
\begin{itemize}
  \item $T = f_L + f_R + f_U + f_D$
  \item $H = f_R - f_L$
  \item $V = f_U - f_D$
\end{itemize}

Напишите программу, которая по значениям $T$, $H$ и $V$ вычислит
соответствующие значения $f_L$, $f_R$, $f_U$ и $f_D$. 

\InputFile

В единственной строке ввода записаны три целых числа $T$, $H$ и $V$
($-4 \cdot 10^8\le T \le 4 \cdot 10^8$,
$-2 \cdot 10^8\le H,\,V \le 2 \cdot 10^8$.

\OutputFile

Выведите четыре целых числа, являющиеся допустимыми значениями сил:
$f_L$, $f_R$, $f_U$ и $f_D$.
Если решений несколько, выведите любое. Гарантируется, что
хотя бы одно решение существует.

\Example

\begin{example}
\exmp{
0 10 0
}{
99999990 100000000 -99999995 -99999995
}%
\end{example}

\end{problem}
