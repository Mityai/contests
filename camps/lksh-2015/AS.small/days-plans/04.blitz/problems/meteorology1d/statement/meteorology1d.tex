% Author: Ivan Kazmenko (classic)
% Text author: Ivan Kazmenko
% Origin: 20110404 - SPbSU Spring 2011 Train 2 (110328) - Data Structures
\begin{problem}{Одномерная метеорология}
{\textsl{стандартный ввод}}{\textsl{стандартный вывод}}
%{meteorology1d.in}{meteorology1d.out}
{2 секунды}{256 мебибайт}{}

Некоторое одномерное царство имеет форму отрезка.
Система координат устроена так, что начало царства имеет координату $0$,
а конец "--- координату $n$.
Для удобства царство поделено на $N$ провинций, пронумерованных числами
$1$, $2$, $\ldots$, $n$ так, что провинция с номером $k$ начинается
в точке с координатой $k - 1$, а заканчивается в точке с координатой $k$.

Сейчас в этом царстве идёт снег. Однократное выпадение снега происходит так.
Прилетает туча, которую можно охарактеризовать параметрами $(a, b, c)$.
Такая туча имеет форму отрезка, покрывающего провинции
с номерами от $a$ до $b$, включительно.
Снег выпадает, и во всех этих провинциях уровень снега вырастает
ровно на $c$ сантиметров, а туча исчезает.

Стихия так разгулялась, что никто не выходит убирать снег.
Время от времени царь велит метеорологам посчитать суммарный уровень снега
в какой-то части царства.
Каждый такой приказ можно охарактеризовать параметрами $(d, e)$.
Часть царства, фигурирующая в этом приказе, имеет форму отрезка,
покрывающего в точности провинции с номерами от $d$ до $e$, включительно.
Суммарный уровень считается как сумма уровней снега в сантиметрах во всех
провинциях этой части царства.

Изначально снега нет ни в одной провинции.
Зная последовательность событий, помогите царским метеорологам правильно
ответить на все вопросы царя.

\InputFile

В первой строке заданы через пробел два целых числа $n$ и $m$ "--- размер
царства и количество событий, соответственно
($1 \le n \le 10^6$, $0 \le m \le 10^5$).
В следующих $m$ строках описаны события в порядке их следования.
Если событие описывает выпадение снега, оно задано в форме
\t{snow $a$ $b$ $c$} ($1 \le a \le b \le n$, $1 \le c \le 10^7$).
Если же событие описывает царский приказ, оно задано в форме
\t{sum $d$ $e$} ($1 \le d \le e \le n$).

\OutputFile

В ответ на каждый приказ царя выведите одно число "--- суммарный уровень
снега в заданной части царства.
Ответы выводите в порядке следования приказов.

\Examples

\begin{example}
\exmp{
5 4
snow 1 5 1
sum 1 5
snow 2 5 2
sum 3 4
}{
5
6
}%
\exmp{
4 7
sum 2 4
snow 1 2 4
sum 2 4
snow 2 3 3
sum 2 4
snow 3 3 2
sum 2 4
}{
0
4
10
12
}%
\end{example}

\end{problem}
