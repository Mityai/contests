%Origin: 201111?? - Leningrad Regional School Olympiad, District Tour
%Author: Ivan Kazmenko
%Description Author: Ivan Kazmenko
%Tests Author: Ivan Kazmenko

\gdef\thisproblemauthor{Иван Казменко}
\begin{problem}{Одно число на месте}
{\textsl{стандартный ввод}}{\textsl{стандартный вывод}}
%{one.in}{one.out}
{2 секунды}{256 мебибайт}{}

Как известно, последовательность из $n$ целых чисел называется перестановкой,
если каждое из чисел $1$, $2$, $\ldots$, $n$ встречается в ней ровно один раз.
Например, последовательности $1$~$2$~$3$ и $2$~$3$~$4$~$1$ "--- перестановки,
а последовательности $1$~$1$ и $2$~$3$~$4$ "--- нет.

Число в перестановке считается стоящим на своём месте, если номер этого
числа, считая с начала перестановки, совпадает с ним самим.
Например, в перестановке $1$~$2$~$3$ каждое число стоит на своём месте,
а в перестановке $2$~$3$~$4$~$1$ ни одно число не стоит на своём месте:
$1$ стоит на четвёртом месте, $2$ "--- на первом, $3$ "--- на втором,
а $4$ "--- на третьем.

Для заданного натурального числа $n$ выведите любую перестановку длины $n$,
в которой на своём месте стоит ровно одно число,
или выясните, что такой перестановки не существует.

\InputFile

В первой строке входного файла задано натуральное число $n$
($1 \le n \le 27$).

\OutputFile

Выведите в выходной файл ровно $n$ целых чисел, образующих перестановку.
В выведенной перестановке ровно одно число должно стоять на своём месте.
При выводе числа следует разделять пробелами.

Если требуемая перестановка не существует, выведите в выходной файл
вместо перестановки одно число \t{-1}.
Если правильных ответов несколько, можно вывести любой из них.

\Examples

\begin{example}%
\exmp{
4
}{
2 3 1 4
}%
\exmp{
3
}{
3 2 1
}%
\end{example}

\Explanations

В первом примере четвёрка стоит на четвёртом месте,
остальные числа "--- не на своих местах.
Это не единственный правильный ответ: например,
в перестановке $1$~$3$~$4$~$2$ на своём месте тоже стоит ровно одно число.

Во втором примере двойка стоит на втором месте,
остальные числа "--- не на своих местах.
Это снова не единственный правильный ответ: например,
в перестановке $2$~$1$~$3$ на своём месте тоже стоит ровно одно число.

\end{problem}
