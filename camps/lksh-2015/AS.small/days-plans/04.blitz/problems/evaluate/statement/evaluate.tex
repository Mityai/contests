% Author: Ivan Kazmenko
% Text author: Ivan Kazmenko
% Origin: 20090306 - Lisiy Nos Spring 2009 Training Session, Elder Group

\begin{problem}{Вычисление значения}
{\textsl{стандартный ввод}}{\textsl{стандартный вывод}}
%{evaluate.in}{evaluate.out}
{2 секунды}{256 мебибайт}{}

Выведите значение заданного арифметического выражения.

\InputFile

В первой строке входного файла задано выражение, состоящее из
чисел, скобок и знаков бинарных операций. Каждое число в выражении "---
это одна цифра от `\t{0}' до `\t{9}', включительно.
Скобки бывают открывающие (`\t{(}') и закрывающие (`\t{)}').
Операции задаются символами `\t{+}', `\t{-}', `\t{*}' и `\t{/}';
знак умножения не может быть опущен.
Гарантируется, что заданное выражение математически корректно,
и результаты всех промежуточных операций "--- целые числа,
не превышающие по модулю $10\,000$.
Выражение не содержит каких-либо других символов, в частности,
пробелов. Длина выражения не меньше $1$ и не больше $1000$ символов.

Учтите, что операции с одинаковым приоритетом при отсутствии скобок
выполняются слева направо. Например, выражение $a + b + c$ вычисляется
как $(a + b) + c$.

\OutputFile

В первой строке выходного файла выведите одно число "--- значение
заданного выражения.

\Examples

\begin{example}
\exmp{
4*8-1*3
}{
29
}%
\exmp{
(5+5)/(2+3)
}{
2
}%
\end{example}

\end{problem}
