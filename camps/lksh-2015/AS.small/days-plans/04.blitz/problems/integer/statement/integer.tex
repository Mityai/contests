%Origin: 201111?? - Leningrad Regional School Olympiad, District Tour
%Author: Ivan Kazmenko
%Description Author: Ivan Kazmenko
%Tests Author: Ivan Kazmenko

\gdef\thisproblemauthor{Иван Казменко}
\begin{problem}{Целые точки}
{\textsl{стандартный ввод}}{\textsl{стандартный вывод}}
%{integer.in}{integer.out}
{2 секунды}{256 мебибайт}{}

Точку на координатной плоскости будем называть целой,
если обе её координаты "--- целые числа.
К примеру, точки $(0, 0)$ и $(-4, 7)$ "--- целые,
а точки $(-1, 0.5)$ и $(\frac{1}{3}, \sqrt{2})$ "--- нет.

Сколько целых точек содержит заданный отрезок на плоскости?

\InputFile

В первой строке входного файла заданы два числа $x_1$ и $y_1$ "---
координаты одного конца отрезка.
Во второй строке заданы два числа $x_2$ и $y_2$ "---
координаты другого конца отрезка.
Числа в каждой строке разделены пробелами.
Все заданные координаты "--- целые числа, не превосходящие по модулю
$1\,000\,000\,000$.
Гарантируется, что заданные две точки не совпадают.

\OutputFile

Выведите в выходной файл количество целых точек на заданном отрезке.
Обратите внимание, что концы отрезка тоже учитываются.

\Examples

\begin{example}%
\exmp{
2 1
4 1
}{
3
}%
\exmp{
0 0
5 7
}{
2
}%
\end{example}

\Explanations

В первом примере целые точки "--- $(2, 1)$, $(3, 1)$ и $(4, 1)$.

Во втором примере целые точки "--- только концы отрезка $(0, 0)$ и $(5, 7)$.

\end{problem}
