%
% Зимние всероссийские школьные учебно-тренировочные сборы по информатике
% понедельник, 3 июля 2012 года
% Источник: Дальневосточный четвертьфинал ACM ICPC NEERC 2011
%
\begin{problem}{Дно}
{\textsl{стандартный ввод}}{\textsl{стандартный вывод}}
%{тест}{ответ}
{2 секунды}{256 мебибайт}{}

Институт исследования подводных глубин проводит исследование морского дна.

Исследуемый участок дна представляется прямоугольной решёткой. Ось
$x$ направлена с запада на восток, ось $y$ "--- с юга на север.

Автоматизированный батискаф производит погружение. У него есть программа,
которая позволяет ему исследовать прямоугольный регион со сторонами,
параллельными осям, заданный координатами юго-западного и северо-восточного
углов.

Исходно планировалось совершить одно погружение с координатами
$(a_x, a_y) - (b_x, b_y)$.

\emph{Внезапно} в игру вступил представитель министерства обороны,
который сообщил, что в регионе $(c_x, c_y) - (d_x, d_y)$
расположен секретный военый склад, куда батискаф пускать не велено.

Было решение разбить регион исследования на меньшие прямоугольные
регионы, для каждого из которых совершить отдельное погружение.
Регионы не могут пересекать друг друга, а также запрещённый регион.
Регионы должны покрывать весь исходный регион $(a_x, a_y) - (b_x, b_y)$,
за исключением запрещённого региона.

Напишите программу, которая выберет минимальное по количеству
множество регионов.

\InputFile

В единственной строке ввода записано через пробел восемь целых чисел:
$a_x$, $a_y$, $b_x$, $b_y$, $c_x$, $c_y$, $d_x$ и $d_y$
($-10^9 \le a_x < b_x \le 10^9$,
$-10^9 \le a_y < b_y \le 10^9$,
$-10^9 \le c_x < d_x \le 10^9$,
$-10^9 \le c_y < d_y \le 10^9$).

\OutputFile

В первой строке выведите минимальное число регионов $M$.
В следующих $M$ строках выведите через пробел четвёрки целых чисел
$x_i$, $y_i$, $u_i$ и $v_i$ "--- координаты юго-восточного и
северо-западного углов.
Если решений несколько, выведите любое.

\Example

\begin{example}
\exmp{
-10 -10 10 10
0 0 20 20
}{
2
-10 -10 10 0
-10 0 0 10
}%
\end{example}

\end{problem}
