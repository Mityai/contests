%Origin: 20111127 - PhML 30 Olympiad, District Tour
%Author: Ivan Kazmenko

\gdef\thisproblemauthor{Иван Казменко}
\begin{problem}{Постройка стены}
{\textsl{стандартный ввод}}{\textsl{стандартный вывод}}
%{wall.in}{wall.out}
{2 секунды}{256 мебибайт}{}
%{\textsl{В этой задаче нужно найти количество способов построить стену.}}

Строитель Василий строит игрушечную стену для детской площадки.
Стена должна состоять из одного ряда кирпичей и иметь $n$ дециметров в длину.
У Василия есть неограниченное количество кирпичей $k$ типов.
Кирпич первого типа имеет длину $1$ дециметр, второго "--- $2$ дециметра,
\ldots, $k$-го типа "--- $k$ дециметров.
Кроме того, кирпичи одного типа покрашены в один цвет,
а кирпичи различных типов "--- в разные цвета.
Каждый кирпич можно использовать только целиком.
Кирпичи в стене должны идти вплотную друг к другу.

Василий привык ответственно относиться к своей работе.
Вот и сейчас, прежде чем построить стену, он задумался:
как будет выглядеть новая стена, какого цвета сделать каждый её дециметр?
В частности, Василий хотел бы знать количество различных раскрасок,
которые могут получиться.

Сколько всего способов построить стену длиной ровно $n$ дециметров?
Два способа считаются различными, если существует дециметр стены,
который при этих способах постройки будет иметь разный цвет.

\InputFile

В первой строке входного файла заданы два целых числа $n$ и $k$ "---
желаемая длина стены в дециметрах и количество типов кирпичей, соответственно
($1 \le k, n \le 30$).

\OutputFile

В первой строке выходного файла выведите одно целое число "--- количество
способов построить стену длины $n$ дециметров, имея $k$ типов кирпичей.

\Examples

\begin{example}%
\exmp{
3 2
}{
3
}%
\exmp{
4 3
}{
7
}%
\exmp{
5 1
}{
1
}%
\exmp{
2 3
}{
2
}%
\end{example}

\Explanations

Для удобства условимся, что кирпичи первого типа "--- красного цвета (\t{К}),
кирпичи второго типа "--- синего цвета (\t{С}),
а кирпичи третьего типа "--- зелёного цвета (\t{З}).
Будем записывать раскраску стены в виде строки из $n$ букв,
каждая из которых отвечает за соответствующий дециметр стены.

В первом примере доступны только первые два типа кирпичей.
Стена может иметь вид
\t{ККК} (три красных кирпича),
\t{КСС} (красный, а за ним синий) или
\t{ССК} (синий, а за ним красный).

Во втором примере могут получиться следующие семь раскрасок стены:
\t{КККК}, \t{ККСС}, \t{КССК}, \t{ССКК}, \t{СССС}, \t{КЗЗЗ} и \t{ЗЗЗК}.

В третьем примере доступны только кирпичи первого типа.
Единственная возможная раскраска стены "--- \t{ККККК}.

В четвёртом примере стена может иметь вид \t{КК} или \t{CC}.
Третий тип кирпичей использовать невозможно: ни один кирпич длины $3$
не поместится в стену длины $2$.

%\PartialSolutions
%
%Решение, правильно работающее при ограничениях
%$1 \le k, n \le 10$,
%получит не менее $40$ баллов.

\end{problem}
