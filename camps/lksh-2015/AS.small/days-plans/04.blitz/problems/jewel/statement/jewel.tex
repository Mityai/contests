%
% Зимние всероссийские школьные учебно-тренировочные сборы по информатике
% понедельник, 3 июля 2012 года
% Источник: Дальневосточный четвертьфинал ACM ICPC NEERC 2011
%
\begin{problem}{Драгоценности}
{\textsl{стандартный ввод}}{\textsl{стандартный вывод}}
%{тест}{ответ}
{2 секунды}{256 мебибайт}{}

Недавно исследователи обнаружили новый вид гигантских кальмаров, живущих
на дне глубоких озёр.

Кальмары, в большинстве случаев, не заинтересованы в общении с людьми.
Однако они интересуются различными драгоценностями, в частности
бриллиантами, которые непросто найти под водой.
Они готовы взамен предложить широкий выбор высококачественных жемчужин.

Сделка была заключена.
Каждый день происходит обмен $N$ различных видов драгоценностей,
некоторые уходят под воду, другие "--- обратно.

Кальмарская торговая комиссия установила следующие правила:
\begin{itemize}
  \item Каждая драгоценность помещается в отдельный контейнер.
  \item Драгоценности одного типа должны быть помещены в одинаковые
    контейнеры.
  \item Торговля не должна изменять уровень воды в озере, то есть
    общий объём всех контейнеров, которые передаются под воду,
    должен быть равен общему объёму всех контейнеров, перемещаемых
    из-под воды.
\end{itemize}

Напишите программу, которая найдёт объём контейнера для каждого
вида драгоценностей, чтобы удовлетворить требования комиссии.

\InputFile

В первой строке записано целое число $N$ "--- количество видов
драгоценностей.
Во второй строке записано $N$ целых чисел $a_i$, где
$a_i > 0$ означает, что $a_i$ драгоценностей типа $i$ перемещаются
под воду озера, а $a_i < 0$ означает, что $\left|a_i\right|$
драгоценностей типа $i$ перемещаются из озера на поверхность.

Гарантируется, что $2 \le N \le 10^5$, $1 \le |a_i| \le 10^5$,
$2 \times \min (pos,\,neg) \ge \max (pos,\,neg)$,
где $pos$ "--- количество положительных значений среди $a_i$, а $neg$ "---
количество отрицательных значений.

\OutputFile

Выведите $N$ целых чисел $b_{i}$ ($1 \le b_i \le 10^{12}$) "---
размеры контейнеров для каждого вида драгоценностей.
Если решений несколько, выведите любое.

\Example

\begin{example}
\exmp{
3
1 2 -3
}{
3 3 3
}%
\end{example}

\end{problem}
