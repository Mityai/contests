%Origin: 201111?? - Leningrad Regional School Olympiad, District Tour
%Author: Ivan Kazmenko
%Description Author: Ivan Kazmenko
%Tests Author: Ivan Kazmenko

\gdef\thisproblemauthor{Иван Казменко}
\begin{problem}{Сумма девяти квадратов}
{\textsl{стандартный ввод}}{\textsl{стандартный вывод}}
%{nine.in}{nine.out}
{2 секунды}{256 мебибайт}{}

Как известно, целое число $a$ называется точным квадратом, если
существует такое целое число $b$, что $a$ является квадратом $b$,
то есть $b^2 = b \cdot b = a$.
Например, $16$ "--- точный квадрат, поскольку $4 \cdot 4 = 16$.
Напротив, $10$ "--- не точный квадрат, так как равенство $b^2 = 10$
неверно ни для какого целого числа $b$.

Представьте заданное во входном файле натуральное число $n$
в виде суммы девяти точных квадратов неотрицательных целых чисел.

\InputFile

В первой строке входного файла задано натуральное число $n$
($1 \le n \le 1\,000\,000\,000$).

\OutputFile

Выведите в выходной файл ровно девять неотрицательных целых чисел:
$a_1$,~$a_2$,~$\ldots$,~$a_9$.
Эти числа должны быть такими, что
$$a_1^2 + a_2^2 + a_3^2 + a_4^2 + a_5^2 + a_6^2 + a_7^2 + a_8^2 + a_9^2 =
 n\text{.}$$
При выводе числа следует разделять пробелами.
Порядок чисел не имеет значения.
Если правильных ответов несколько, можно вывести любой из них.

\Examples

\begin{example}%
\exmp{
5
}{
1 2 0 0 0 0 0 0 0
}%
\exmp{
9
}{
1 1 1 1 1 1 1 1 1
}%
\end{example}

\Explanations

В первом примере
$1^2 + 2^2 + 0^2 + 0^2 + 0^2 + 0^2 + 0^2 + 0^2 + 0^2 = 1 + 4 = 5$.
Это не единственный правильный ответ: например,
ответ \t{1 1 1 1 1 0 0 0 0} тоже подходит.

Во втором примере
$1^2 + 1^2 + 1^2 + 1^2 + 1^2 + 1^2 + 1^2 + 1^2 + 1^2 = 9 \cdot 1 = 9$.
Это не единственный правильный ответ: например,
ответ \t{0 0 0 0 0 0 0 0 3} тоже подходит.

\end{problem}
