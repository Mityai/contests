%
% Зимние всероссийские школьные учебно-тренировочные сборы по информатике
% понедельник, 3 июля 2012 года
% Источник: Дальневосточный четвертьфинал ACM ICPC NEERC 2011
%
\begin{problem}{Ихтиология}
{\textsl{стандартный ввод}}{\textsl{стандартный вывод}}
%{тест}{ответ}
{2 секунды}{256 мебибайт}{}

Учёные Института ихтиолингвистики изучают язык рыб. Они выяснили, что некоторые
рыбы могут издавать специфичные звуки, и сопоставили им буквы латинского
алфавита. Теперь они получили запись звука моря и с помощью специального ПО
преобразовали их в строку из букв.

Предполагатеся, что одна подстрока из букв может иметь некоторое определённое
значение в языке рыб, то есть окажется словом.
Поэтому они хотят узнать, сколько раз она могла встречаться в записи.

Напишите программу, которая по строкам $T$ и $W$ найдёт минимальное и максимальное
количество непересекающихся вхождений $W$ в $T$.

Например, если $W$ = <<\t{abab}>> и $T$ = <<\t{ababbbabababab}>>,
строку $T$ можно
интерпретировать как <<\t{(abab)bb(abab)(abab)}>>, то есть найти три вхождения,
а можно "--- как <<\t{(abab)bbab(abab)ab}>>, то есть найти два вхождения.

\InputFile

В первой строке записана непустая строка $W$.
Во второй строке записана непустая строка $T$.
Гарантируется, что $1 \le \mathrm{length} (W) \le 100$ и
$1 \le \mathrm{length} (T) \le 1000$.
Строки $W$ и $T$ состоят из строчных букв английского алфавита.

\OutputFile

Выведите два целых числа: минимальное и максимальное количество вхождений
$W$ в $T$.

\Example

\begin{example}
\exmp{
a
b
}{
0 0
}%
\end{example}

\end{problem}
