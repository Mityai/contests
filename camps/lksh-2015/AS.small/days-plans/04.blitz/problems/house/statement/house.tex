%Origin: 20111127 - PhML 30 Olympiad, District Tour
%Author: Ivan Kazmenko

\gdef\thisproblemauthor{Иван Казменко}
\begin{problem}{Цветной дом}
{\textsl{стандартный ввод}}{\textsl{стандартный вывод}}
%{house.in}{house.out}
{2 секунды}{256 мебибайт}{}
%{\textsl{В этой задаче нужно выяснить цвет крыши дома по цветам полов
%и потолков всех этажей.}}

На окраине одного небольшого городка стоит недавно построенный цветной дом.
В этом доме $n$ этажей и $n + 1$ горизонтальная перегородка.
Самая нижняя перегородка "--- это пол первого этажа,
вторая снизу перегородка служит потолком первого этажа и полом второго,
и так далее.
Последняя перегородка "--- потолок верхнего этажа "--- является также
крышей дома.
Каждая горизонтальная перегородка этого дома со всех сторон покрашена
в какой-то один цвет.

В этом доме также есть лифт с $n$ кнопками.
Каждая кнопка отправляет лифт на какой-то определённый этаж,
разные кнопки "--- на разные этажи.
Однако кнопки не подписаны, поэтому какая из них ведёт на какой этаж,
неизвестно.

Исследуя цветной дом, мальчик Коля оказался в лифте.
Он произвёл серию из $n$ наблюдений: нажал по одному разу каждую из $n$
кнопок, дождался, пока лифт приедет на соответствующий этой кнопке этаж
и откроет двери, после чего записал себе в блокнот цвет пола и потолка
на этаже, на который он попал.
Для краткости Коля записывал вместо цветов целые числа так,
что одному и тому же цвету соответствовало одно и то же число,
а различным цветам "--- разные числа.

Придя домой, Коля внезапно понял, что забыл, какого цвета крыша
у цветного дома.
Он помнит только, что цвет пола самого нижнего этажа и цвет крыши различны.
Пользуясь колиными записями, помогите ему восстановить цвет крыши.

\InputFile

%\vskip -2pt

В первой строке входного файла задано целое число $n$ ($1 \le n \le 100$) "---
количество этажей в доме.
Следующие $n$ строк содержат по два целых числа каждая "--- записи
колиных наблюдений.
Первое из чисел в строке "--- цвет пола какого-то этажа, а второе "---
цвет потолка этого этажа.
Соседние числа в этих строках разделены пробелами.
Поскольку кнопки в лифте не подписаны, порядок этажей в этих записях
может быть произвольным.
Номера цветов, использованные Колей, лежат в диапазоне от $1$
до $10\,000$ включительно.

Гарантируется, что колины записи корректны, то есть существует цветной дом,
в котором описанная серия из $n$ наблюдений могла дать такие результаты.

\OutputFile

%\vskip -2pt

В первой строке выходного файла выведите одно целое число,
соответствующее цвету крыши цветного дома.

\Examples

%\vskip -2pt

%\begin{exampledouble}%
%\exmpd{
%2
%1 2
%2 3
%}{
%3
%}{
%3
%2 7
%3 5
%5 2
%}{
%7
%}%
%\end{exampledouble}

\begin{example}
\exmp{
2
1 2
2 3
}{
3
}%
\exmp{
3
2 7
3 5
5 2
}{
7
}%
\end{example}

\Explanations

%\vskip -2pt

В первом примере дом состоит из двух этажей.
Пол первого этажа имеет цвет $1$, потолок первого и пол второго "--- цвет $2$,
а потолок второго и крыша дома "--- цвет $3$.

Во втором примере дом состоит из трёх этажей.
Пол первого этажа имеет цвет $3$, потолок первого и пол второго "--- цвет $5$,
потолок второго и пол третьего "--- цвет $2$.
Наконец, потолок третьего этажа и крыша дома имеют цвет $7$.
Этажи в записях Коли перечислены в следующем порядке:
третий, первый, второй.

%\PartialSolutions
%
%\vskip -2pt
%
%Решение, правильно работающее при ограничениях
%$1 \le n \le 10$, диапазоне номеров цветов от $1$ до $100$ включительно
%и дополнительном условии о том,
%что все $n + 1$ цветов горизонтальных перегородок различны,
%получит не менее $40$ баллов.

\end{problem}
