%Origin: 20111127 - PhML 30 Olympiad, District Tour
%Author: Ivan Kazmenko

%\gdef\thisproblemauthor{Иван Казменко}
\begin{problem}{Улитка на ступеньке}
{\textsl{стандартный ввод}}{\textsl{стандартный вывод}}
%{step.in}{step.out}
{2 секунды}{256 мебибайт}{}
%{\textsl{В этой задаче нужно вычислить минимальную длину пути улитки.}}

{

\begin{figure}
\includegraphics{step.1}
\end{figure}

Рассмотрим систему, состоящую из плоской горизонтальной поверхности и
бесконечной в обе стороны ступеньки высоты $h$.
Ступенька состоит из двух частей: вертикальной и горизонтальной.
В пространстве введена трёхмерная прямоугольная декартова система координат.
Ось $Ox$ совпадает с нижним краем вертикальной части ступеньки,
а сама вертикальная часть ступеньки находится в плоскости $Oxz$.
Поверхность под ступенькой "--- часть плоскости,
для которой $z = 0$ и $y \le 0$.
Наконец, горизонтальная часть ступеньки "--- часть плоскости,
для которой $z = h$ и $y \ge 0$.
Общий вид системы показан на рисунке.

}

Улитка находится на плоской горизонтальной поверхности неподалёку от ступеньки
в точке $(x_s, y_s, z_s)$.
На ступеньке в точке $(x_t, y_t, z_t)$ растёт вкусная травинка,
до которой улитка хочет добраться.
При вычислениях следует считать улитку и травинку точками в пространстве.
Улитка ползёт по любой из трёх поверхностей "--- подножию ступеньки,
её вертикальной части и её верхней части "--- с одинаковой скоростью,
поэтому ей важна только длина проделанного пути.

Какое минимальное расстояние придётся преодолеть улитке, чтобы добраться
до травинки?

\InputFile

В первой строке входного файла заданы три целых числа $x_s$, $y_s$ и $z_s$ "---
координаты улитки ($-1000 \le x_s \le 1000$, $-1000 \le y_s < 0$, $z_s = 0$).
Во второй строке заданы три целых числа $x_t$, $y_t$ и $z_t$ "---
координаты травинки ($-1000 \le x_t \le 1000$, $0 < y_t \le 1000$, $z_t = h$).
Высота ступеньки $h$ "--- целое число такое, что $1 \le h \le 1000$.
Соседние числа в каждой строке разделены пробелами.

\OutputFile

В первой строке выходного файла выведите одно число "---
минимальное расстояние, которое потребуется преодолеть улитке,
чтобы добраться до травинки.
Ответ может быть не точным, но должен отличаться от правильного
не более чем на $10^{-4}$.

\Example

\begin{example}%
\exmp{
0 -2 0
0 4 3
}{
9
}%
\end{example}

\Explanation

{

\begin{figure}
\includegraphics{step.2}
\end{figure}

%\begin{wrapfigure}{r}{0.5\thelinewidth}
%\begin{wrapfigure}{r}{0.30\thelinewidth}
%\vskip -10pt
%\includegraphics{step.2}
%\end{wrapfigure}

В примере высота ступеньки равна $3$.
Кратчайший путь для улитки "--- проследовать из точки $(0, -2, 0)$
сначала по горизонтальной поверхности в точку $(0, 0, 0)$,
затем по вертикальной части ступеньки в точку $(0, 0, 3)$,
а после этого по горизонтальной части ступеньки в точку $(0, 4, 3)$.
Кратчайшее расстояние равно $2 + 3 + 4 = 9$.
Путь улитки показан на рисунке.

Допускается вывод ответа с десятичной точкой (\t{9.0}, \t{9.000000000}),
а также экспоненциальная форма вывода (\t{9.0E0}).
Помните, что выведенный ответ должен отличаться от правильного
не более чем на \t{0.0001}.
К примеру, ответ \t{8.99995} на тест из примера будет считаться правильным,
а ответ \t{9.000123} "--- не будет.

}

%\PartialSolutions
%
%Решение, правильно работающее при дополнительном ограничении $x_s = x_t = 0$,
%получит не менее $40$ баллов.

\end{problem}
