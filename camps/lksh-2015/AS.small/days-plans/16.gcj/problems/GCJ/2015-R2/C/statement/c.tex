% Перевод: Сергей Копелиович
% Источник: GCJ 2015, Round 2, Problem C

\begin{problem}{Англофранцузский словарь}
{stdin}{stdout}
{1 секунда}{256 мегабайт}{}

Родители Вани дома говорят по-французски и по-английски. Он слышит много сло, но не сегда знает, из какого именно языка то или иное слово.
Ваня записал $N$ предложений (текстов).
Про первое Ваня уверен, что оно состоит целиком из английских слов.
А про второе предложение Ваня уверен, что оно состоит целиком из французских слов.
А про оставшиеся Ваня не уверен, но точно знает, что каждое предложение или целиком английское, или целиком французское.
Заметим, что у английского и французского много общего, поэтому некоторые слова могут встречаться в обоих языках.
Помогите Ване так распределить $N$ предложений по языковой принадлежности, чтобы количество слов одновременно
входящих и во французский, и в английский было минимально возможным.

\InputFile

Первая строка содержит количество тестов $T$, далее следует описание $T$ тестов.
Каждый тест содержит $N$ и $N$ предложений, каждое на отедельной строке.
Предложения состоят из слов, слова из маленьких букв \t{a-z}.

\OutputFile

Для каждого теста выведите на отдельной строке одно целое число -- минимальное число слов, которые одновременно и английские, и французские.

\Scoring

$1 \le T \le 25$\\
Каждое слово состоит из не более чем 10 букв.
Два первых предложения содержат не более 1000 слов каждое.
Оставшиеся предложения содержат не более 10 слов каждое.

\begin{itemize}
  \item {\bf Простая подзадача:} $2 \le N \le 20$ 
  \item {\bf Сложная подзадача:} $2 \le N \le 200$ 
\end{itemize}

\Examples

\begin{example}
\exmp{
4
2
he loves to eat baguettes
il aime manger des baguettes
4
a b c d e
f g h i j
a b c i j
f g h d e
4
he drove into a cul de sac
elle a conduit sa voiture
il a conduit dans un cul de sac
il mange pendant que il conduit sa voiture
6
adieu joie de vivre je ne regrette rien
adieu joie de vivre je ne regrette rien
a b c d e
f g h i j
a b c i j
f g h d e
}{
1
4
3
8
}%
\end{example}

\end{problem}

%In Case #1, Elliot knows for sure that the first sentence is in English and the second is in French, so there is no ambiguity; the only word that must be in both English and French is "baguettes".
%In Case #2, the last two sentences could either be: English English, English French, French English, or French French. The second of those possibilities is the one that minimizes the number of words common to both languages; that set turns out to be d, e, i, and j.
