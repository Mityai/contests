% Перевод: Павел Маврин
% Источник: GCJ 2015, Round 2, Problem B

\begin{problem}{Бассейн}
{stdin}{stdout}
{1 секунда}{256 мегабайт}{}

У вас есть большой детский бассейн, в который нужно набрать воду, чтобы дети могли играть в нем.

Есть $n$ кранов с водой. Из крана $i$ течет вода с температурой $c_i$ градусов, и скоростью $r_i$ литров в минуту. Первоначально все краны закрыты. Каждый кран можно открыть и закрыть только один раз, это не занимает дополнительного времени. Несколько кранов могут быть открыты одновременно.

Вы хотите набрать в бассейн ровно $v$ литров воды с температурой ровно $x$ градусов, и сделать это как можно быстрее. Если вы будете действовать оптимально, через сколько секунд бассейн будет заполнен?

В этой задаче мы будем считать, что при объединении воды объемом $v_0$ и температурой $x_0$ с водой объемом $v_1$ и температурой $x_1$, получается вода объемом $v_0 + v_1$ и температурой $\frac{v_0 x_0 + v_1 x_1}{v_0 + v_1}$. Например, смешав 5 литров воды с температурой 10 градусов с 10 литрами воды с температурой 40 градусов, получится 15 литров воды с температурой 30 градусов. Также будем считать, что вода не нагреваться и не охлаждаться в процессе, кроме как в результате смешивания с другой водой.

\InputFile

Первая строка содержит количество тестов $t$, далее следует описание $t$ тестов.
Первая строка каждого теста содержит три числа: натуральное $n$ и вещественные $v$ и $x$.

Следующие $n$ строк содержат по два вещественных числа $r_i$ и $c_i$: скорость потока и температура воды, текущей из $i$-го крана. 

Все вещественные числа даны с точностью до четырех знаков.

\OutputFile

Для каждого теста выведите на отдельной строке минимальное количество секунд, необходимое для того, чтобы заполнить бассейн. Если это невозможно сделать, выведите строку \texttt{IMPOSSIBLE}.

Ответ будет считаться правильным, абсолютная или относительная погрешностью не превышает $10^{-6}$. 

\Scoring

$1\le t\le 100$.
$0.1\le x\le 99.9$.
$0.1\le c_i\le 99.9$.

\begin{itemize}
  \item {\bf Простая подзадача} \\
	$1\le n\le 2$.
	$0.0001\le v\le 100.0$.
	$0.0001\le r_i\le 100.0$.
  \item {\bf Сложная подзадача} \\
	$1\le n\le 100$.
	$0.0001\le v\le 10000.0$.
	$0.0001\le r_i\le 10000.0$.
\end{itemize}

\Examples

\begin{example}
\exmp{
6
1 10.0000 50.0000
0.2000 50.0000
2 30.0000 65.4321
0.0001 50.0000
100.0000 99.9000
2 5.0000 99.9000
30.0000 99.8999
20.0000 99.7000
2 0.0001 77.2831
0.0001 97.3911
0.0001 57.1751
2 100.0000 75.6127
70.0263 75.6127
27.0364 27.7990
4 5000.0000 75.0000
10.0000 30.0000
20.0000 50.0000
300.0000 95.0000
40.0000 2.0000
}{
50.0000000
207221.843687375
IMPOSSIBLE
0.500000000
1.428034895
18.975332068
}%
\end{example}

\end{problem}
