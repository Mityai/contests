% Перевод: Сергей Копелиович
% Источник: GCJ 2015, Round 2, Problem B

\begin{problem}{Очередь в парикмахерскую}
{stdin}{stdout}
{1 секунда}{256 мегабайт}{}

Вы хотите постричься, и уже долго сидите в очереди в модной парикмахерской.
В парикмахерской $B$ мастеров-парикмахеров занумерованных числами от $1$ до $B$. 
У $k$-го парикмахера на стрижку посетителя всегда уходит ровно $M_k$ минут.
Парикмахер не начинает обслуживать нового посетителя, пока не закончит работу с предыдущим.
Как только парикмахер заканчивает работу над одним посетителем, он может мгновенно переключиться на следующего.
В каждый момент, если есть свободные парикмахеры, первый клиент в очереди выберет среди свободных парикмахера с наименьшим номером и пойдёт стричься.
Если свободных парикмахеров нет, вся очередь ждёт.
Вы $N$-й в очереди. К какому из парикмахеров вы попадёте?

\InputFile

Первая строка содержит количество тестов $T$, далее следует описание $T$ тестов.
Каждый тест состоит из двух строк. Первая содержит два числа разделённых пробелами $B$, $N$
Количество парикмахеров и ваш номер в очереди. 
Во второй строке содержится $N$ чисел $M_i$.

\OutputFile

Для каждого теста выведите на отдельной строке целое число от 1 до $K$ -- номер вашего парикмахера.

\Scoring

$1 \le T \le 100$.
$1 \le N \le 10^9$.

\begin{itemize}
  \item {\bf Простая подзадача} \\
	$2 \le B \le 5$ \\
	$0 \le M_k \le 25$
  \item {\bf Сложная подзадача} \\
	$2 \le B \le 1000$ \\
	$0 \le M_k \le 100000$
\end{itemize}

\Examples

\begin{example}
\exmp{
3
2 4
10 5
3 12
7 7 7
3 8
4 2 1
}{
1
3
1
}%
\end{example}

\Note

В первом примере, вы 4-й по порядку, парикмахеры 1 и 2 требуют 10 и 5 минут, соответственно, на стрижку. 
Когда магазин открывается, 1-й посетитель сразу же выбирает между парикмахерами 1 и 2, и, конечно, выберет из них того, у кого номер меньше, 1-го.
Второй посетитель сразу же пойдёт ко 2-му парикмахеру. 
Третий посетитель будет ждать, т.к. совободных парикмахеров нет.
Через 5 минут 2-й парикмахер освободится и начнёт обслуживать 3-го посетителя.
Через 10 минут, оба парикмахера 1 и 2 закончат работу, вы следующий, и вы выберите среди них того, у кого номер меньше, 1-го.

\end{problem}
