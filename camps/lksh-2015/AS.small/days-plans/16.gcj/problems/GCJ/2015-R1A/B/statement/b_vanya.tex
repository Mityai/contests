Problem


Вы пришли в парикмахерскую и встали в очередь. В данный момент времени работают $B$ парикмахеров. Известно, что $i$-ый парикмахер делает стрижку 
за $M_i$ минут. Как только он освобождается он может сразу же начать обслуживать следующего клиента. Если в один момент времени освободились несколько  парикмахеров, то следующий посетитель из очереди идёт к тому, у которого меньше номер. 
Вы стоите $N$-ым человеком в очереди. Определите к какому парикмахеру вы пойдёте стричься.
В начальный момент времени парикмахерская только открылась и все парикмахеры свободны.


%You are waiting in a long line to get a haircut at a trendy barber shop. The shop has B barbers on duty, and they are numbered 1 through B. It always takes the kth barber exactly Mk minutes to cut a customer's hair, and a barber can only cut one customer's hair at a time. Once a barber finishes cutting hair, he is immediately free to help another customer.

%While the shop is open, the customer at the head of the queue always goes to the lowest-numbered barber who is available. When no barber is available, that customer waits until at least one becomes available.

%You are the Nth person in line, and the shop has just opened. Which barber will cut your hair?

Input

Первая строка содержит количество тестов $T$. 

Каждый тест состоит из двух строк. Первая содержит два числа разделённых пробелами $B$, $N$
Количество парикмахеров и ваш номер в очереди. 
Во второй строке содержится $N$ чисел $M_i$.
%The first line of the input gives the number of test cases, T. T test cases follow; each consists of two lines. The first contains two space-separated integers B and N -- the number of barbers and your place in line. The customer at the head of the line is number 1, the next one is number 2, and so on. The second line contains M1, M2, ..., MB.

Output

Для каждого теста выведите единственное число, номер парикмахера у которого вы будите стричься.

%For each test case, output one line containing "Case #x: y", where x is the test case number (starting from 1) and y is the number of the barber who will cut your hair.

Limits

1 ≤ T ≤ 100.
1 ≤ N ≤ 109.
Small dataset

1 ≤ B ≤ 5.
1 ≤ Mk ≤ 25.
Large dataset

1 ≤ B ≤ 1000.
1 ≤ Mk ≤ 100000.
Sample


Input 
 	
Output 
 
3
2 4
10 5
3 12
7 7 7
3 8
4 2 1

Case #1: 1
Case #2: 3
Case #3: 1

In Case #1, you are the fourth person in line, and barbers 1 and 2 take 10 and 5 minutes, respectively, to cut hair. When the shop opens, the first customer immediately has the choice of barbers 1 and 2, and she will choose the lowest-numbered barber, 1. The second customer will immediately be served by barber 2. The third customer will wait since there are no more free barbers. After 5 minutes, barber 2 will finish cutting the second customer's hair, and will serve the third customer. After 10 minutes, both barbers 1 and 2 will finish; you are next in line, and you will have the choice of barbers 1 and 2, and will choose 1.
