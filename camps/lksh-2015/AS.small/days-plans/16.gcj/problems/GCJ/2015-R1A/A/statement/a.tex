% Перевод: Сергей Копелиович
% Источник: GCJ 2015, Round 2, Problem A

\begin{problem}{Кайлин и грибы}
{stdin}{stdout}
{1 секунда}{256 мегабайт}{}

Кайлин любит грибы. Очень.

Мы проводим эксперимент. Перед Кайлин стоит одна тарелка, а Бартеломей время от времени накладывает ей ещё грибов.
Каждые 10 секунд сперва Бартеломей накладывает сколько-то грибов, затем
Кайлин съедает сколько-то грибов, но не более чем есть на тарелке, и наконец происходит отметка в журнале -- сколько грибов осталось на тарелке.

Ваша задача -- по отметкам в журнале определить минимальное количество грибов, котороые за всё прошедшее время съела Кайлин.
При этом Кайлин могла действовать по одной из двух стратегий:
\begin{enumerate}
  \item Она в любой момент времени есть любое число грибов.
  \item Она ела грибы с постоянной скоростью.
\end{enumerate}

Например, если отметки в журнале были 10 5 15 5 (первая отметка сделана в начальный момент времени, затем 3 раза после каждых 10 секунд),
то по первой стратегии Кайлин съела бы как минимум 15 грибов, а по второй 25 (т.к. скорость поедания не менее 10).

\InputFile

Первая строка содержит количество тестов $T$, далее следует описание $T$ тестов.
Каждый тест содержит $N$ и $N$ целых чисел $m_i$, разделённых пробелами; отметки в журналев начале и после каждого 10-секундного перерыва.

\OutputFile

Для каждого из тестов выведите строку ``$y$ $z$'',
где $y$ -- минимальное число грибов, которые Кайлин съела бы используя первую стратегию,
а $z$ -- минимальное число грибов, которые Кайлин может съесть, используя вторую стратегию.

\Scoring

\begin{itemize}
  \item {\bf Простая подзадача} \\
	$2 \le N \le 10$ \\
	$0 \le m_i \le 100$
  \item {\bf Сложная подзадача} \\
	$2 \le N \le 1000$ \\
	$0 \le m_i \le 10000$
\end{itemize}

\Examples

\begin{example}
\exmp{
4
4
10 5 15 5
2
100 100
8
81 81 81 81 81 81 81 0
6
23 90 40 0 100 9
}{
15 25
0 0
81 567
181 244
}%
\end{example}

\end{problem}
