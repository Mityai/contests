\begin{problem}{Набор строк}{typing.in}{typing.out}{2 секунды}{256 мегабайт}

В Инновационном Отделе НИИ Исследований Данных Строк разработана клавиатура для внутреннего
пользования, облегчающая набор строк огромной длины. Кроме обычных клавиш, 
соответствующих маленьким латинским буквам, на клавиатуре
есть еще $n$ функциональных клавиш $F_1$, \ldots, $F_n$, 
соответствующих заданным строкам из словаря $S_1$, \ldots $S_n$.
При нажатии такой клавиши $F_i$ строка $S_i$ загружается во внутреннюю память клавиатуры. В
каждый момент времени в памяти может находиться не более
одной строки из словаря.

Кроме того,
в клавиатуру встроен графический манипулятор <<Кыш>>, с помощью которого легким
движением руки можно ввести любую подстроку находящейся в памяти строки.

Вася занимается исследованием эффективности данного нововведения. Для этого
ему требуется написать программу, которая будет вычислять минимальное
необходимое количество действий (нажатий и использований
<<Кыш>>) для ввода данной строки $S$. В момент начала ввода строки память пуста. 

Например, если требуется ввести строку ``\texttt{abacaba}'', а в словаре есть строки
``\texttt{baba}'' и ``\texttt{caca}'', то это можно сделать за четыре действия --- нажать $F_1$, 
выбрать манипулятором подстроку ``\texttt{aba}'', затем нажать `\texttt{c}', и
опять выбрать манипулятором подстроку ``\texttt{aba}''. Если бы нужно было
набрать с таким
словарем ``\texttt{bacababa}'', то это можно сделать за пять действий: `\texttt{b}', $F_2$,
``\texttt{aca}'', $F_1$, ``\texttt{baba}''.


\InputFile

В первой строке входного файла задано число $n$ ($1\le n\le 50$). В последующих $n$
строках заданы $S_i$, составленные из не более чем 500 символов.
В последней строке вводится непустая строка $S$, длина которой не превосходит
$100\,000$. Все символы строк --- маленькие латинские буквы.

\OutputFile

Выведите минимальное необходимое количество действий.

\Example

\begin{example}
\exmp{
2
baba
caca
abacaba
}{
4
}%
\exmp{
2
baba
caca
bacababa
}{
5
}%
\end{example}

\end{problem}