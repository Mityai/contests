\begin{problem}{Перестановки}{permutation.in}{permutation.out}{2 секунды}{64 мегабайта}

%Author: Andrew Lopatin
%Text Author: Vitaly Valtman

Вася выписал на доске в каком-то порядке все числа от 1 по $N$, каждое число ровно по одному разу. Количество чисел оказалось
довольно большим, поэтому Вася не может окинуть взглядом все числа. Однако ему надо всё-таки представлять эту последовательность, поэтому
он написал программу, которая отвечает на вопрос~--- сколько среди чисел, стоящих на позициях с $x$ по $y$, по величине лежат
в интервале от $k$ до $l$. Сделайте то же самое.

\InputFile

В первой строке лежит два натуральных числа --- $1\le N\le 100\,000$ --- количество чисел, которые выписал Вася и 
$1\le M\le 100\,000$ --- количество вопросов, которые Вася хочет задать программе.
Во второй строке дано $N$ чисел --- последовательность чисел, выписанных Васей. 
Далее в $M$ строках находятся описания вопросов. Каждая строка содержит четыре целых числа $1\le x\le y\le N$ и
$1\le k\le l\le N$.

\OutputFile

Выведите $M$ строк, каждая должна содержать единственное число~--- ответ на Васин вопрос.

\Example

\begin{example}%
\exmp{
4 2
1 2 3 4
1 2 2 3
1 3 1 3
}{
1
3
}%
\end{example}

\end{problem}
