\begin{problem}{Обратная инверсия-2}{invers2.in}{invers2.out}{2 секунды}{256 мебибайт}{U}

\emph{Таблицей инверсий} для перестановки $A=(a_1, a_2, \ldots, a_n)$ чисел
$\{1,2,\ldots,N\}$ 
называется массив $X=(x_i)_{1\leq i\leq N}$, в котором
на $i$-м месте стоит количество элементов, больших $i$,
но стоящих левее, чем $i$, т.е 
$x_i=\hbox{число таких\ }j',\hbox{что} j'<j, a_{j'}>a_j=i$.

Например, таблицей инверсий для перестановки $(2, 5, 1, 3, 4)$ будет
$(2, 0, 1, 1, 0)$, а для перестановки $(6, 1, 3, 7, 5, 4, 2)$ ---
$(1, 5, 1, 3, 2, 0, 0)$.

Обратной перестановкой $A^{-1}$ к перестановке $A$ называется такая перестановка
чисел, что на $i$-м месте в $A^{-1}$ стоит номер места, на котором стоит
элемент, равный $i$, в перестановке $A$. 

Например, для перестановки $(2, 5, 1, 3, 4)$ обратной
будет $(3, 1, 4, 5, 2)$ (т.к. 1 стоит на третьем месте, 2 --- на первом,
3 --- на четвертом, 4 --- на пятом, а 5 --- на втором), 
а для перестановки $(2, 7, 3, 6, 5, 1, 4)$ обратной будет
$(6, 1, 3, 7, 5, 4, 2)$.

Ваша задача --- по таблице инверсий перестановки $A$ посчитать таблицу инверсий
обратной перестановки $A^{-1}$.

\InputFile

Файл состоит ровно из $N$ чисел, разделенных пробелами и переводами
строки, задающих таблицу инверсий перестановки $A$. Число
$N$ находится в пределах от 1 до $\mathbf{262\,144}$.

\OutputFile

Выведите $N$ целых чисел, разделенных пробелами --- таблицу инверсий
для обратной перестановки.

\Example

\begin{example}

\exmp{
2 0 1 1 0
}{
1 3 0 0 0
}%
\exmp{
5 0 1 3 2 1 0
}{
1 5 1 3 2 0 0
}%
\end{example}

\end{problem}
