% Problem author: Sergey Kopeliovich
% Изначально задача создана для сборов в Харькове в феврале 2012 года

\begin{problem}{Count Online}{countonline.in}{countonline.out}{2 секунды}{256 мегабайт}{}

Вам дано множество точек на плоскости.

Нужно уметь отвечать на два типа запросов:

$\circ$ \t{? $x_1$ $y_1$ $x_2$ $y_2$} --- сказать, сколько точек лежит в прямоугольнике $[x_1..x_2] \times [y_1..y_2]$.
Точки на границе и в углах тоже считаются. $x_1 \le x_2$, $y_1 \le y_2$.

$\circ$ \t{+ $x$ $y$} --- добавить в множество точку \t{(x + res \% 100, y + res \% 101)}.
Где \t{res} --- ответ на последний запрос вида \t{?}, а \t{\%} --- операция взятия по модулю.

\InputFile

Число точек $N$ ($1 \le N \le 50\,000$).
Далее $N$ точек.
Число запросов $Q$ ($1 \le Q \le 100\,000$).
Далее $Q$ запросов.
Все координаты от $0$ до $10^9$.

\OutputFile

Для каждого запроса $\t{GET}$ одно целое число --- количество точек внутри прямоугольника.

\Example

\begin{example}
\exmp{
5
0 0
1 0
0 1
1 1
1 1
9
? 0 1 1 2
+ 1 2
+ 2 2
? 1 0 2 2
? 0 0 0 0
+ 3 3
? 3 3 3 3
? 4 3 4 3
? 4 4 5 5
}{
3
3
1
0
0
3}%
\end{example}

\Note

На самом деле добавлялись точки \t{(4, 5)}, \t{(5, 5)}, \t{(4, 4)}.

\end{problem}
