\documentclass[12pt,a4paper,oneside]{article}

\usepackage{cmap}
\usepackage[T2A]{fontenc}
\usepackage[utf8]{inputenc}
\usepackage[english,russian]{babel}
\usepackage[russian]{olymp}
\usepackage{graphicx}
\usepackage{amsmath,amssymb}
\usepackage{epigraph}
\usepackage[russian]{hyperref}
\usepackage{color}
\usepackage{lastpage}
\usepackage{import}
\usepackage{verbatim}

\renewcommand{\t}{\texttt}
\renewcommand{\le}{\leqslant}
\renewcommand{\ge}{\geqslant}

\binoppenalty=10000
\relpenalty=10000
\exhyphenpenalty=10000

\newcommand{\ProblemLabel}{undefined}
\newcommand{\ProblemTL}{undefined}
\newcommand{\ProblemML}{undefined}
\newcommand{\ProblemName}{undefined}

\def\O{\mathcal{O}}

\def\probl#1#2#3#4#5{
  \renewcommand{\ProblemName}{#2}
  \renewcommand{\ProblemLabel}{#3}
  \renewcommand{\ProblemTL}{#4}
  \renewcommand{\ProblemML}{#5}
  %\input #2.tex
  \import{../problems/#1/#2/statement/}{#2}
  %\input ../problems/#1/#2/statement/#2.tex
}
\def\problfix#1#2#3#4#5{
  \renewcommand{\ProblemName}{#2}
  \renewcommand{\ProblemLabel}{#3}
  \renewcommand{\ProblemTL}{#4}
  \renewcommand{\ProblemML}{#5}
  \input ../problems/#1/#2/statement/#2.tex
}
          
\newcommand{\Section}[1]{
  \hbox{\hspace{1em}}
  \vspace*{-2.5em}
  \section*{#1}
  \addcontentsline{toc}{section}{#1}
  \vspace*{-0.5em}
}

\contest
{ЛКШ.2015.Июль.AS.День 8: dynamic connectivity}%
{Судиславль, Берендеевы Поляны}%
{14 июля 2015, вторник}%

%\sectionfont{\fontsize{8}{8}\selectfont}

\definecolor{dkgreen}{rgb}{0,0.6,0}
\definecolor{brown}{rgb}{0.5,0.5,0}

\def\compact{
  \setlength{\parskip}{-5pt}
  \setlength{\itemsep}{5pt}
}

\begin{document}

\vspace*{-2em}
\tableofcontents

\vspace*{1em}

\noindent \underline{\hbox to 1\textwidth{{ } \hfil{ } \hfil{ } }}

\pagebreak

\probl{2015-07}{treap}{A}{0.5 sec}{256 mb}     % - Храним мальчиков в некотором порядке, у каждого есть неотрицательная интересность. Запросы: изменить интересность, соединить две группы мальчиков, разъединить группу, вернуть всех мальчиков, с положительной интересностью.
\probl{2015-03}{eulertour}{B}{0.5 sec}{256 mb} % - Граф всегда является лесом. Нужно обрабатывать запросы Link, Cut, IsConnected(a, b)
\probl{2012-04}{connect}{C}{0.5 sec}{256 mb}   % - Offline. Ребра добавляются, удаляются, нужно говорить кол-во компонент связности. Есть решение за O(KlogK).

\end{document}
