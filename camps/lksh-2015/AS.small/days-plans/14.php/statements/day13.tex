\documentclass[12pt,a4paper,oneside]{article}

\usepackage{cmap}
\usepackage[T2A]{fontenc}
\usepackage[utf8]{inputenc}
\usepackage[english,russian]{babel}
\usepackage[russian]{olymp}
\usepackage{graphicx}
\usepackage{amsmath,amssymb}
\usepackage{epigraph}
\usepackage[russian]{hyperref}
\usepackage{color}
\usepackage{lastpage}
\usepackage{import}
\usepackage{verbatim}

\renewcommand{\t}{\texttt}
\renewcommand{\le}{\leqslant}
\renewcommand{\ge}{\geqslant}

\binoppenalty=10000
\relpenalty=10000
\exhyphenpenalty=10000

\newcommand{\ProblemLabel}{undefined}
\newcommand{\ProblemTL}{undefined}
\newcommand{\ProblemML}{undefined}
\newcommand{\ProblemName}{undefined}

\def\O{\mathcal{O}}

\def\probl#1#2#3#4#5{
  \renewcommand{\ProblemName}{#2}
  \renewcommand{\ProblemLabel}{#3}
  \renewcommand{\ProblemTL}{#4}
  \renewcommand{\ProblemML}{#5}
  %\input #2.tex
  \import{../problems/#1/#2/statement/}{#2}
  %\input ../problems/#1/#2/statement/#2.tex
}
\def\problfix#1#2#3#4#5{
  \renewcommand{\ProblemName}{#2}
  \renewcommand{\ProblemLabel}{#3}
  \renewcommand{\ProblemTL}{#4}
  \renewcommand{\ProblemML}{#5}
  \input ../problems/#1/#2/statement/#2.tex
}
          
\newcommand{\Section}[1]{
  \hbox{\hspace{1em}}
  \vspace*{-2.5em}
  \section*{#1}
  \addcontentsline{toc}{section}{#1}
  \vspace*{-0.5em}
}

\contest
{ЛКШ.2015.Июль.AS.День 13: unknown language day}%
{Судиславль, Берендеевы Поляны}%
{22 июля 2015, среда}%

%\sectionfont{\fontsize{8}{8}\selectfont}

\definecolor{dkgreen}{rgb}{0,0.6,0}
\definecolor{brown}{rgb}{0.5,0.5,0}

\def\compact{
  \setlength{\parskip}{-5pt}
  \setlength{\itemsep}{5pt}
}

\begin{document}

\vspace*{-2em}
\tableofcontents

\vspace*{1em}

\noindent \underline{\hbox to 1\textwidth{{ } \hfil{ } \hfil{ } }}

\pagebreak


\probl{2011-11}{numbers}{A}{0.5 sec}{256 mb}      % - AX + BY = 1 (2 цикла for)
\probl{2011-10b}{perm}{B}{0.5 sec}{256 mb}        % - [перебор] все перестановки
\probl{2011-03}{dominoes}{C}{0.5 sec}{256 mb}     % - кол-во замощений доминошками, WxH <= 50 (time = 2^{50/2}, дописать условие!)
\probl{2011-11}{spaces}{D}{0.5 sec}{256 mb}       % - удалить все лишние пробелы (начальные, конечные, парные промежуточные)
\probl{2011-10}{greatest}{E}{0.5 sec}{256 mb}     % - наибольшая буква в строке (строки)
\probl{2011-04}{nenokku_easy}{F}{0.5 sec}{256 mb} % -  Халява (техника работы со сторками)
\probl{2011-10}{catlover}{G}{0.5 sec}{256 mb}     % - число треугольников в графе. N <= 1000, M <= 30 000.
\probl{2011-10}{tree}{H}{0.5 sec}{256 mb}         % - является ли граф деревом (N <= 100)
\probl{2014-03}{distance}{I}{0.5 sec}{256 mb}     % - N <= 10^5, M <= 2*10^5, W <= 100, найти расстояние от s до t
\probl{2012-07}{cover}{J}{0.5 sec}{256 mb}        % - min кол-вом отрезков покрыть [0..M]
\probl{2012-07}{ejudge}{K}{0.5 sec}{256 mb}       % - N <= 10^5, x <= 10^5, сортировка по 3-м параметрам (x, y, индекс)
\probl{2010-12}{distance1}{L}{0.5 sec}{256 mb}    % - Расстояние от точки до прямой
\probl{2011-10}{product}{M}{0.5 sec}{256 mb}      % - произведение двух (int64)
\probl{2012-08}{recover}{N}{0.5 sec}{256 mb}      % - Сколько способов заменить вопросики так, чтобы скобочная посл-ть стала правильной (N <= 10 000)
\probl{2015-02}{milliarder}{O}{0.5 sec}{256 mb}   % - map<string, int> human_id, city_id; map<int, int> human2city, city2money; set<pair<int,int>> money_and_city;
\probl{2014-07}{kthstr}{P}{0.5 sec}{256 mb}       % - Добавить строку в бор. Вернуть k-ю строку.

\end{document}
