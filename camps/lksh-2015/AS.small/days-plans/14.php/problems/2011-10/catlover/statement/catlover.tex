\begin{problem}{Любители Кошек}{catlover.in}{catlover.out}{0.1 секунда}{256 мегабайт}{1E}

В университетском клубе любителей кошек зарегистрировано $n$ членов.
Естественно, что некоторые из членов клуба знакомы друг с другом.
Нужно сосчитать, сколькими способами можно выбрать из них троих, которые могли бы
свободно общаться (то есть, любые два из которых знакомы между собой).

\InputFile

В первой строке входного файла заданы числа $n$ и $m$ ($1\le n\le 1\,000$, $1\le m\le 30\,000$),
где $m$ обозначает общее число знакомств. В последующих $m$ строках идут пары чисел
$a_i$ $b_i$, обозначающие, что $a_i$ знаком с $b_i$. Информация
об одном знакомстве может быть записана несколько раз, причем
даже в разном порядке (как $(x, y)$, так и $(y, x)$).

\OutputFile

В выходной файл необходимо вывести количество способов выбрать троих попарно знакомых
друг с другом людей из клуба.

\Example

\begin{example}
\exmp{
3 3
1 2
2 3
3 1
}{
1
}%
\end{example}

\end{problem}
