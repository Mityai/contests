\begin{problem}{K-я строка}{kthstr.in}{kthstr.out}{2 секунды}{64 мегабайта}

Реализуйте структуру данных, которая поддерживает следующие операции:

\begin{itemize}\compact
  \item добавить в словарь строку $S$;
  \item найти в словаре $k$-ю строку в лексикографическом порядке.
\end{itemize}

Изначально словарь пуст.

\InputFile

Первая строка входного файла содержит натуральное число $N$ "---
количество команд ($N \le 10^5$). Последующие $N$ строк содержат по
одной команде каждая.

Команда записывается либо в виде числа $k$, либо в виде строки $S$, которая
может состоять только из строчных латинских букв.
Гарантируется, что при запросе $k$-й строки она существует.
Также гарантируется, что сумма длин всех добавляемых строк не превышает $10^5$.

\OutputFile

Для каждого числового запроса $k$ выходной файл должен содержать $k$-ю 
в~лексикографическом порядке строчку из словаря на момент запроса.
Гарантируется, что суммарная длина строк в выходном файле не превышает $10^5$.

\Examples

\begin{example}
\exmp{7
pushkin
lermontov
tolstoy
gogol
gorkiy
5
1
}{tolstoy
gogol
}%
\end{example}

\end{problem}

