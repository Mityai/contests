% Задача Ненокку только количество запросов не больше $50$, а размер входного файла не больше $1$ кб.

\begin{problem}{Ненокку (простая)}{nenokku\_easy.in}{nenokku\_easy.out}{1 секунда}{64 мегабайта}

Очень известный автор не менее известной книги решил написать 
продолжение своего произведения. 
Он писал все свои книги на компьютере, подключенном к интернету. 
Из-за такой неосторожности мальчику Ненокку удалось получить доступ к еще ненаписанной книге. 
Каждый вечер мальчик залазил на компьютер писателя и записывал на свой компьютер 
новые записи. Ненокку, записав на свой компьютер очередную главу, 
заинтересовался, а использовал ли хоть раз писатель слово ``книга''. 
Но он не любит читать книги (он лучше полазает в интернете), и поэтому 
он просит вас узнать есть ли то или иное слово в тексте произведения. 
Но естественно его интересует не только одно слово, а достаточно много.

\InputFile

В каждой строчке входного файла записано одна из двух записей.

\begin{enumerate}
\item \t{? <слово>} (<слово> - это набор не более 50 латинских символов);
\item \t{A <текст>} (<текст> - это набор не более 1024 латинских символов).
\end{enumerate}

1 означает просьбу проверить существование подстроки <слово> в произведение.

2 означает добавление в произведение <текст>.

Число запросов --- не более 30. Входной файл содержит не более 1 килобайта. 

\OutputFile

Выведите на каждую строчку типа 1 ``\t{YES}'', если существует 
подстрока <слово>, и ``\t{NO}'' в противном случае. Не следует 
различать регистр букв.

\Example

\begin{example}%
\exmp{%
? love
? is
A Loveis
? love
? WHO
A Whoareyou
? is
}{%
NO
NO
YES
NO
YES
}%
\end{example}

\end{problem}
