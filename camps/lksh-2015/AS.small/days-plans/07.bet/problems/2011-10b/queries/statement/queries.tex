\begin{problem}{Для любителей статистики}{queries.in}{queries.out}{1 секунда}{64 мегабайта}

% Author: Alexander Ipatov
% Prepared by: Alexander Ipatov
% Text by: Alexander Ipatov

Вы никогда не задумывались над тем, сколько человек за год перевозят 
трамваи города с десятимиллионным населением, в котором каждый третий житель
пользуется трамваем по два раза в день?

Предположим, что на планете Земля $n$ городов, в которых есть трамваи.
Любители статистики подсчитали для каждого из этих городов, сколько человек
перевезено трамваями этого города за последний год. Из этих данных была составлена 
таблица, в которой города были отсортированы по алфавиту. Позже выяснилось, что
для статистики названия городов несущественны, и тогда их просто заменили числами
от 1 до $n$. Поисковая система, работающая с этими данными,
должна уметь быстро отвечать на вопрос, есть ли среди городов с номерами от $l$ до $r$
такой, что за год трамваи этого города перевезли ровно $x$ человек.
Вам предстоит реализовать этот модуль системы.

\InputFile
В первой строке дано целое число $n$, $0 < n < 70\,000$. 
В следующей строке приведены статистические данные в виде
списка целых чисел через пробел, $i$-е число в этом списке~--- 
количество человек, перевезенных за год трамваями $i$-го города. 
Все числа в списке положительны и не превосходят $10^9-1$. 
В третьей строке дано количество запросов $q$, $0 < q < 70\,000$.
В следующих $q$ строках перечислены запросы. Каждый запрос~--- это тройка целых чисел
$l$, $r$ и $x$, записанных через пробел
($1 \leqslant l \leqslant r \leqslant n$,  $0 < x < 10^9$).

\OutputFile
Выведите строку длины $q$, в которой 
$i$-й символ равен \texttt{1}, если ответ на $i$-й запрос утвердителен, 
и \texttt{0} в противном случае.

\Example
\begin{example}
\exmp{
5
123 666 314 666 434
5
1 5 314
1 5 578
2 4 666
4 4 713
1 1 123
}{
10101
}%
\end{example}

\end{problem}
