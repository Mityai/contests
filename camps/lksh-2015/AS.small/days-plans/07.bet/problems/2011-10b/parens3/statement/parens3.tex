% Author: Ivan Kazmenko
% Text author: Ivan Kazmenko
% Origin: 20080424 - SPb DTU Training, prepared for 20080423
\begin{problem}{Три типа скобок}
{parens3.in}{parens3.out}
{2 секунды}{64 мегабайта}

\newcommand {\T} {\mathcal{T}}
Определим по индукции множество $\T$ {\it правильных скобочных
последовательностей из трёх типов скобок}:
\renewcommand {\t} {\texttt}

\begin {itemize}
 \item $\varepsilon \in \T$ (пустая строка)
 \item $A \in \T \Rightarrow \t{(}A\t{)} \in \T$
 \item $A \in \T \Rightarrow \t{[}A\t{]} \in \T$
 \item $A \in \T \Rightarrow \t{\{}A\t{\}} \in \T$
 \item $A \in \T, \, B \in \T \Rightarrow AB \in \T$
\end {itemize}

Пусть теперь $\T_n$ --- это множество правильных скобочных
последовательностей из $2 n$ символов --- $n$ открывающих и
$n$ закрывающих скобок.

Упорядочим элементы множества $\T_n$ лексикографически с некоторым
порядком символов `\t{(}', `\t{)}', `\t{[}', `\t{]}', `\t{\{}' и `\t{\}}'.

По данным числам $n$ и $p$, а также порядку, заданному на скобках,
найдите $p$-ый в этом порядке элемент множества $\T_n$.

\InputFile

В первой строке входного файла заданы через пробел два целых числа $n$ и $p$
($0 \leqslant n \leqslant 20$, $0 \leqslant p \leqslant 9 \cdot 10^{18}$).
Скобочные последовательности нумеруются с нуля.

Во второй строке записаны шесть символов ---
`\t{(}', `\t{)}', `\t{[}', `\t{]}', `\t{\{}' и `\t{\}}' --- в некотором
порядке. Их порядок задаёт лексикографический порядок на множестве $\T_n$.

\OutputFile

В первой строке выходного файла выведите $2 n$ символов без пробелов ---
$p$-ю правильную скобочную последовательность длины $2 n$ из трёх
типов скобок.

Если для данного $n$ не существует $p$-я правильная скобочная
последовательность, выведите в первой строке ``\t{N/A}''.

\Examples

\begin{example}
\exmp{
1 0
()\}[\{]
}{
()
}%
\exmp{
1 1
()\}[\{]
}{
[]
}%
\exmp{
1 2
()\}[\{]
}{
\{\}
}%
\exmp{
1 3
()\}[\{]
}{
N/A
}%
\end{example}

\end{problem}
