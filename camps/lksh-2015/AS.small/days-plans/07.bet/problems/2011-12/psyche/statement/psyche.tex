% Источник = летние школьные сборы к межнару, 2005-й год

\begin{problem}{Психотренинг}{psyche.in}{psyche.out}{2 секунды}{64 мегабайта}

На очередном психологическом тренинге $n$ участников сборов играют в 
занимательную игру. Участники игры рассаживаются по кругу и получают
номера от 1 до $n$ против часовой стрелки. После этого главный 
психолог отсчитывает против часовой стрелки $k$-го участника игры, 
начиная с первого. Этот участник выходит из круга и может идти
на ужин. А остальные продолжают участие в тренинге. Главный психолог
отсчитывает еще $k$ участников, начиная со следующего после выбывшего.
Участник, который оказался $k$-ым, тоже покидает тренинг, и так далее.

Участники сборов решили сесть в круг таким образом, чтобы один
вредный тип пошел ужинать последним. Для этого они хотят установить,
какой номер он должен для этого получить. Помогите им.

\InputFile

Входной файл содержит два целых числа: $n$ и $k$ ($1 \le n \le 10^{18}$,
$1 \le k \le 1000$).


\OutputFile

Выведите в выходной файл одно число~--- номер участника, который
пойдет на ужин последним.

\Example

\begin{example}
\exmp{
5 3
}{
4
}%
\end{example}

\end{problem}
                                          