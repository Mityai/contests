\begin{problem}{Том Сойер и его друзья}{paint.in}{paint.out}{1 секунда}{64 мегабайта}

Друзья Тома Сойера по очереди красят забор разными красками. Каждый из них красит несколько идущих подряд секций забора
в определенный цвет, при этом используемые цвета могут повторяться. Новая краска ложится поверх старой.
Для каждой краски вычислите количество секций, которые будут покрашены этой краской после того, как все
друзья закончат работу.

\InputFile

В первой строке входного файла содержатся два целых числа: $N$ ($1 \le N \le 10^9$) и $K$ ($1 \le K \le 50\,000$)~---
количество секций в заборе и количество различных красок соответственно.

Во второй строке содержится единственное число $M$ ($0 \le M \le 50\,000$)~--- количество друзей Тома Сойера.

Далее следуют $M$ строк: в $i$-ой строке содержится информация о работе друга, который красил забор $i$-ым 
по счету, а именно 3 целых числа $c_i$, $l_i$, $r_i$ ($1 \le c_i \le K$, $1 \le l_i \le r_i \le N$)~---
номер краски, которую использовал $i$-й друг, номер первой и номер последней покрашенной секции соответственно.

\OutputFile

Выведите в единственную строку выходного файла $K$ целых чисел: $i$-ое число должно быть равно количеству секций, покрашенных $i$-й
краской.

\Example                

\begin{example}
\exmp{
5 3
4
1 3 4
2 4 5
3 2 3
1 5 5
}{
1 1 2
}%
\exmp{
5 3
3
1 1 5
2 2 4
1 3 3
}{
3 2 0
}%
\end{example}

\end{problem}
